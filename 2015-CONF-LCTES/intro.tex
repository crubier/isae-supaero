%---------------------------------------
%The interactive part of a system is important since it becomes more and more complex. The introduction of several complex digital interactive (input/output) devices led to 
%\begin{itemize}
%\item an important part of code allowed to the development of interface
%\item complexity due to the growing et of interaction possibilities offered thanks to the availability of different interaction devices that offer several services
%\end{itemize}
 %Less effort is payed for the design and development of the interactive part of a system compared to the functional part of the same system. 
%In this paper, we advocate 
%--------------------------------------------

In the  last 30 years,  the aerospace domain has  successfully devised
rigorous  languages, methods  and tools  for the  development of  safe
functionally-correct software. In the  same time, interactive software
received  a   very  lower   amount  of  attention.    However,  highly
interactive  Human  Machine  Interfaces  (HMI) are  now  appearing  in
critical  embedded  systems  and   particularly  in  aeronautics:  new
generations of aircraft cockpits  make use of sophisticated electronic
and  digital devices  that  may be  driven by  more  and more  complex
software  applications  endowed  with  a   huge  number  of  lines  of
code. These applications must behave as intended with a high degree of
assurance because  of their  criticality. An  error in  these software
components may have catastrophic consequences.

So  there  is  a  real  stake   to  master  the  development  and  the
implementation of  these critical interactive applications.  The heart
of the problem  is that the standard processes for  the development of
safety critical  software in aeronautics  are not really  suitable for
interactive  software design  for  which more  iterative methods  that
involve end-users  and that mix  design and tests are  still required.
Moreover,  the  stakeholders  that   participate  in  the  design  and
implementation  of these  applications  do not  have  common means  to
express  properly  and  rigorously  the intended  behaviour  of  these
interfaces.

In  this paper  we advocate  devising a  well-defined domain  specific
language  for representing  the behavior  of the  designed interactive
software in  a way that allows,  on the one hand,  system designers to
iterate  on  their designs  before  injecting  them in  a  development
process  and, on  the other  hand,  system developers  to check  their
software against the chosen design.

Section 2 details some motivations to devise such a new DSL. 

