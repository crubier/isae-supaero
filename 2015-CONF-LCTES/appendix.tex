

\subsection{Generic}

Some generic interactions are necessary to make the language useful. Listing~\ref{lst:generic} shows the most important ones.

\begin{lstlisting}[caption=Section of the standard LIDL library regarding generic interactions,label={lst:generic}]
// (if(cond)then(a)else(b)) is similar 
// to the ternary operator cond?a:b in C
interaction
	(if (condition:boolean in)
	then (a:<type>)
	else (b:<type>))
implementing
	-<type>
is
	`native`
	
// ((a) is active) projects (a) into the 
// activation type. It removes the value of 
// (a), only keeping its activation
interaction
	((a:<type> in) is active)
implementing
	activation out
is
	`native`

// ((a) is true) takes a boolean and returns
// an activation which is active if and only
// if (a) is active and true
interaction
	((a:boolean in) is true)
implementing
	activation out
is
	`native`

// (bind(a):(b)) is equivalent to (a), 
// but it also execute the behavior (b) 
// in parallel. (b) can have an effect 
// on (a) by using variables
interaction
	(bind (a:<type> data) : (b:activation out))
implementing
	<type> data
is
	`native`

interaction 
	((a: <type> in) default (b: <type> in))
implementing
	<type> out
is
	(if ((a) is active) then (a) else (b))
\end{lstlisting}

\subsection{Lists}


\begin{lstlisting}[caption=Section of the standard LIDL library regarding lists interactions,label={lst:lists}]
// Get list element
interaction 
	( (a:[<type>]in) [ (index:number in) ] )
implementing
	<type> out
is
	`native`
	
// Set list element
interaction 
	( (a:[<type>]out) [ (index:number in) ] )
implementing
	<type> in
is
	`native`
	
// Compose list
interaction
	([(first:<type>in),(second:<type>in),...,(last:<type>in)])
implementing
	<type> out
is
	`native`
	
	
// Decompose list
interaction
	([(first:<type>out),(second:<type>out),...,(last:<type>out)])
implementing
	<type> in
is
	`native`
\end{lstlisting}



\subsection{Behavior}
What is a behavior ? Very generally, a behavior is some kind of action that is executed at some points in time. This is why we define the \lstinline{Behavior} interface to be equivalent to the \lstinline{activation in} interface. This means that interactions implementing the \lstinline{Behavior} interface receive a flow of \lstinline{activation}. When they receive the \lstinline{active} value, they are effective, when they receive the \lstinline{inactive} value, they are not effective. 

The LIDL standard library defines some useful basic behaviors, as shown on Listing~\ref{lst:behavior}.

\begin{lstlisting}[caption=Section of the standard LIDL library regarding Behaviors,label={lst:behavior}]
interface 
	Behavior 
is
	activation in

interaction 
	((a:<type>out) = (b:<type>in))
implementing
	Behavior
is 
	`native`

interaction 
	(all (a:activation out) (b:activation out))
implementing 
	Behavior
is
	`native`
	
interaction 
	(either (a:activation out) (b:activation out))
implementing 
	Behavior
is
	`native`
	
interaction 
	(always (a:activation out))
implementing 
	Behavior
is
	`native`
	
interaction
	(when (condition:activation in) : (effect:activation out) )
implementing
	Behavior
is
	((effect)=(condition))

interaction 
	(make (a: <type> s) flow)
implementing
	Behavior
is
	((a) = ((new(a)) default (previous(a))))
  
\end{lstlisting}


\subsection{Activation}


\begin{lstlisting}
interaction
	(active)
implementing
	activation out
is
	`native`
\end{lstlisting}



\subsection{Boolean operators}

What is a boolean operator ? Very generally, it is something that takes some parameters and returns a boolean. This is why we define de \lstinline{Boolean} interface to be equivalent to the \lstinline{boolean out} interface. This means that interactions implementing the \lstinline{Boolean} interface return a flow of \lstinline{boolean} which depends on their parameters.

The LIDL standard library defines some useful basic boolean operators, as shown on Listing~\ref{lst:boolean}.

\begin{lstlisting}[caption=The Number expressions definitions of the standard LIDL library,label={lst:boolean}]
interface 
	Boolean 
is
	boolean out

interaction 
	(not (a:boolean in))
implementing 
	Boolean
is
	`native`

interaction 
	((a:boolean in) and (b:boolean in))
implementing 
	Boolean
is
	`native`
	
interaction 
	((a:boolean in) or (b:boolean in))
implementing 
	Boolean
is
	(not((not(a))and(not(b))))
	
interaction 
	((a:boolean in) xor (b:boolean in))
implementing 
	Boolean
is
	(((a)or(b))and(not((a)and(b))))
	
interaction 
	((a:boolean in) nand (b:boolean in))
implementing 
	Boolean
is
	(not((a)and(b))
	
interaction	
	((a:boolean in) nor (b:boolean in))
implementing 
	Boolean
is
	(not((a)or(b))

interaction 
	((a:boolean in) implies (b:boolean in))
implementing 
	Boolean
is
	(not((a)and(not(b))))
	
interaction 
	((a:boolean in) is equivalent to (b:boolean in))
implementing 
	Boolean
is
	(not((a)xor(b)))
	
interaction 
	((a:boolean in) inhibited by (b:boolean in))
implementing 
	Boolean
is
	((a)and(not(b)))
	
interaction 
	((a:<type>in) == (b:<type>in))
implementing
	Boolean
is 
	`native`
	
interaction 
	((a:<type>in) != (b:<type>in))
implementing
	Boolean
is 
	(not((a)==(b)))

\end{lstlisting}


\subsection{Numeric operators}

What is a numeric operator ? Very generally, it is something that takes some parameters and returns a number. This is why we define de \lstinline{Numeric} interface to be equivalent to the \lstinline{number out} interface. This means that interactions implementing the \lstinline{Numeric} interface return a flow of \lstinline{number} which depends on their parameters.

The LIDL standard library defines some useful basic numeric operators, as shown on Listing~\ref{lst:numeric}.

\begin{lstlisting}[caption=The Number expressions definitions of the standard LIDL library,label={lst:numeric}]
interface 
	Numeric 
is
	number out

// opposite
interaction 
	(- (a:number in))
implementing 
	Numeric
is
	`native`

// sum
interaction 
	((a:number in) + (b:number in))
implementing 
	Numeric
is
	`native`
	
// difference
interaction 
	((a:number in) - (b:number in))
implementing 
	Numeric
is
	((a)+(-(b)))

// inverse
interaction
	(inverse (a:number in))
implementing
	Numeric
is
	`native`

// multiplication
interaction 
	((a:number in) * (b:number in))
implementing 
	Numeric
is
	`native`

// division
interaction 
	((a:number in) / (b:number in))
implementing 
	Numeric
is
	((a)*(inverse(b)))
\end{lstlisting}





\subsection{Widgets}

The LIDL standard library defines some widgets, as shown on Listing~\ref{lst:widgets}.

\begin{lstlisting}[caption=A standard widget library,label={lst:widgets}]
////////////////////////////////////////////////
interface
	Label
is
	{
		value : text out
	}

interaction 
	(Label (status: activation in)
	with title (theTitle: text in))
implementing
	Label
is
	(bind(this) : (when (status) : (all
			((this.value)=(theTitle)))
	)))

////////////////////////////////////////////////
interface 
	Gauge
is
	{
		value : number out
	}

interaction 
	(Gauge (status: activation in)
 	with value (theValue: number in))
implementing 
	Gauge
is
	(bind(this) : (when (status) : (all
		((this.value)=(theValue))
	)))
	
////////////////////////////////////////////////
interface 
	Button
is
	{
		value: text out,
		click : activation in	
	}

interaction 
	(Button (status: activation in) 
	with title (theTitle: text in) 
	trigerring (onClick: activation out))
implementing
	Button
is
	(bind(this) : (when (status) : (all
			((this.value)=(theTitle))
			((onClick)=(this.click))
	)))
	
	
    
////////////////////////////////////////////////
interface 	
	SegmentedSwitch
is
	{
		choices : [text] out,
		selection : number in
	}

interaction 
	(SegmentedSwitch (status: activation in) 
	with choices (theChoices: [text] in)
	selecting (theSelection: text out))
implementing
	SegmentedSwitch
is
	(bind(this) : (when (status) : (all
			((this.choices)=(theChoices))
			((theSelection)=
				((theChoices)[(this.selection)]))
	)))
	
////////////////////////////////////////////////
interface 
	Slider
is
	{
		value : number out
		selection : number in,
	}

interaction 
	(Slider (status: activation in)
	with value (theValue: number in)
	selecting (theSelection: number out))
implementing
	Slider
is
	(bind(this) : (when (status) : (all
			((this.value)=(theValue))
			((theSelection)=(this.selection))
	)))


////////////////////////////////////////////////

\end{lstlisting}


Here we put the LIDL source code of the case study and probably the promela code and Event-B code etc. 

