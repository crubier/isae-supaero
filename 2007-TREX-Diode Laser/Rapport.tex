% This file was converted to LaTeX by Writer2LaTeX ver. 0.4
% see http://www.hj-gym.dk/~hj/writer2latex for more info
\documentclass[12pt,twoside]{article}
\usepackage[ascii]{inputenc}
\usepackage[T1]{fontenc}
\usepackage[english,french]{babel}
\usepackage{amsmath,amssymb,amsfonts,textcomp}
\usepackage{color}
\usepackage{calc}
\usepackage{hyperref}
\hypersetup{colorlinks=true, linkcolor=blue, filecolor=blue, pagecolor=blue, urlcolor=blue}
\newcommand\textsubscript[1]{\ensuremath{{}_{\text{#1}}}}
% Outline numbering
\setcounter{secnumdepth}{3}
\renewcommand\thesection{\Roman{section}}
\renewcommand\thesubsection{\Roman{section}.\arabic{subsection}}
\renewcommand\thesubsubsection{\Roman{section}.\arabic{subsection}.\Alph{subsubsection}}
% List styles
\newcommand\liststyleLii{%
\renewcommand\labelitemi{[25CF?]}
\renewcommand\labelitemii{[25CB?]}
\renewcommand\labelitemiii{[25A0?]}
\renewcommand\labelitemiv{[25CF?]}
}
\newcommand\liststyleLiii{%
\renewcommand\labelitemi{[25CF?]}
\renewcommand\labelitemii{[25CB?]}
\renewcommand\labelitemiii{[25A0?]}
\renewcommand\labelitemiv{[25CF?]}
}
\newcommand\liststyleLiv{%
\renewcommand\theenumi{\arabic{enumi}}
\renewcommand\theenumii{\arabic{enumii}}
\renewcommand\theenumiii{\arabic{enumiii}}
\renewcommand\theenumiv{\arabic{enumiv}}
\renewcommand\labelenumi{\theenumi.}
\renewcommand\labelenumii{\theenumii.}
\renewcommand\labelenumiii{\theenumiii.}
\renewcommand\labelenumiv{\theenumiv.}
}
\newcommand\liststyleLv{%
\renewcommand\labelitemi{[25CF?]}
\renewcommand\labelitemii{[25CB?]}
\renewcommand\labelitemiii{[25A0?]}
\renewcommand\labelitemiv{[25CF?]}
}
\newcommand\liststyleLvi{%
\renewcommand\labelitemi{[25CF?]}
\renewcommand\labelitemii{[25CB?]}
\renewcommand\labelitemiii{[25A0?]}
\renewcommand\labelitemiv{[25CF?]}
}
\newcommand\liststyleLvii{%
\renewcommand\labelitemi{[25CF?]}
\renewcommand\labelitemii{[25CB?]}
\renewcommand\labelitemiii{[25A0?]}
\renewcommand\labelitemiv{[25CF?]}
}
\newcommand\liststyleLviii{%
\renewcommand\labelitemi{[25CF?]}
\renewcommand\labelitemii{[25CB?]}
\renewcommand\labelitemiii{[25A0?]}
\renewcommand\labelitemiv{[25CF?]}
}
\newcommand\liststyleLix{%
\renewcommand\labelitemi{[25CF?]}
\renewcommand\labelitemii{[25CB?]}
\renewcommand\labelitemiii{[25A0?]}
\renewcommand\labelitemiv{[25CF?]}
}
% Pages styles (master pages)
\makeatletter
\newcommand\ps@Standard{%
\renewcommand\@oddhead{}%
\renewcommand\@evenhead{}%
\renewcommand\@oddfoot{}%
\renewcommand\@evenfoot{}%
\setlength\paperwidth{20.999cm}\setlength\paperheight{29.699cm}\setlength\voffset{-1in}\setlength\hoffset{-1in}\setlength\topmargin{2cm}\setlength\headheight{12pt}\setlength\headsep{0cm}\setlength\footskip{12pt+0cm}\setlength\textheight{29.699cm-2cm-2cm-0cm-12pt-0cm-12pt}\setlength\oddsidemargin{2cm}\setlength\textwidth{20.999cm-2cm-2cm}
\renewcommand\thepage{\arabic{page}}
\setlength{\skip\footins}{0.101cm}\renewcommand\footnoterule{\vspace*{-0.018cm}\noindent\textcolor{black}{\rule{0.25\columnwidth}{0.018cm}}\vspace*{0.101cm}}
}
\makeatother
\pagestyle{Standard}
\everymath{\displaystyle}
\begin{document}

\bigskip


\bigskip


\bigskip


\bigskip


\bigskip

{\centering
 [Warning: Image not found] 
\par}


\bigskip


\bigskip


\bigskip


\bigskip


\bigskip

{\centering
TREX 1A
\par}


\bigskip

{\centering\itshape
Diode laser
\par}


\bigskip


\bigskip


\bigskip


\bigskip


\bigskip


\bigskip


\bigskip


\bigskip


\bigskip


\bigskip

{\itshape
Encadrant}

DUFFAUT Jo\"el


\bigskip

{\itshape
\'El\`eves}

LECRUBIER Vincent

OLIVIER Solenne

\setcounter{tocdepth}{10}
\renewcommand\contentsname{Sommaire}
\tableofcontents

\bigskip

\clearpage\section[\ G\'en\'eralit\'es]{\ G\'en\'eralit\'es}
\subsection[\ Structure de l'atome]{\ Structure de l'atome}
Un atome est constitu\'e d'\'electrons et d'un noyau. Les \'electrons se
r\'epartissent autour du noyau en fonction de leur \'energie qui ne
peut prendre que des valeurs pr\'ecises. Ces diff\'erentes dispositions
en fonction du niveau d'\'energie sont appel\'ees \'etats de l'atome.

L'\'etat d'\'energie minimale est appel\'e \'etat fondamental ou \'etat
stable. Dans cet \'etat, les \'electrons sont le plus proche du noyau.
L'atome peut passer dans des \'etats d'\'energie sup\'erieure, dits
\'etats excit\'es.

\subsection[\ Les changements d'\'etat de l'atome]{\ Les changements
d'\'etat de l'atome}
\subsubsection[\ L'absorption]{\ L'absorption}
L'atome passe d'un \'etat stable \`a un \'etat excit\'e gr\^ace \`a un
apport d'\'energie. Cette \'energie peut \^etre apport\'ee sous deux
formes :

\liststyleLii
\begin{enumerate}
\item sous la forme d'\'energie cin\'etique (Cons\'ecutive \`a un choc)
\item sous forme \'electromagn\'etique (Rayonnement).
\end{enumerate}
Lorsque cette \'energie est \'electromagn\'etique, elle est transmise
par des photons. Chaque photon a une \'energie $E$ d\'efinie par :

{\centering
 $E=h\cdot \nu =h\cdot {\frac{c}{\lambda }}$ 
\par}

Avec

 $h=6,62\cdot 10^{-34}J\cdot s^{-1}$ constante de Planck

 $\nu $ fr\'equence qui caract\'erise le photon

 $c=2,997\cdot 10^{8}m\cdot s^{-1}$ c\'el\'erit\'e de la lumi\`ere dans
le vide 

 $\lambda $  longueur d'onde du photon


\bigskip

Pour faire passer l'atome d'un \'etat d'\'energie $E_{1}$
\textsubscript{ }\`a un \'etat d'\'energie sup\'erieur \textsubscript{
$E_{2}$ }, le photon doit avoir une \'energie  $E$ v\'erifiant :

{\centering
 $E=E_{2}-E_{1}$ 
\par}


\bigskip

La transition d{\textquotesingle}un \'etat d{\textquotesingle}\'energie 
$E_{1}$ \`a un \'etat  $E_{2}$ ne se fera donc que si
l{\textquotesingle}atome re\c{c}oit un photon de fr\'equence $\nu $
telle que :

{\centering\bfseries
 $E_{2}-E_{2}=h\cdot \nu $ 
\par}

{\centering\bfseries
\begin{minipage}{7.753cm}
{\itshape
[Warning: Draw object ignored]Illustration 1: Ph\'enom\`ene
d{\textquotesingle}absorption}
\end{minipage}
\par}

\subsubsection[\ L'\'emission spontan\'ee ]{\ L'\'emission spontan\'ee }
L'atome passe d'un \'etat excit\'e \`a un niveau d'\'energie plus bas.
Cette chute s'accompagne d'une \'emission d'\'energie qui peut avoir
deux formes :

\liststyleLiii
\begin{enumerate}
\item Thermique (dissipation de chaleur)
\item Rayonnement \'electromagn\'etique : si on passe d'un niveau
$E_{2}$ au niveau plus bas \textsubscript{ $E_{1}$ }, alors un photon
d'\'energie  $E=E_{2}-E_{1}=h\cdot \nu $ sera \'emis.
\end{enumerate}
Le rayonnement \'electromagn\'etique obtenu est incoh\'erent car les
atomes \'emettent ind\'ependamment les uns des autres et dans toutes
les directions. Cependant la fr\'equence est identique pour tous les
photons lib\'er\'es, si l{\textquotesingle}on n\'eglige
l{\textquotesingle}\'etalement en fr\'equence d\^u a
l{\textquotesingle}effet \foreignlanguage{english}{Doppler} caus\'e par
l{\textquotesingle}agitation thermique.

{\centering
\begin{minipage}{7.753cm}
{\itshape
[Warning: Draw object ignored]Illustration 2: Ph\'enom\`ene
d{\textquotesingle}\'emission spontan\'ee}
\end{minipage}
\par}

\subsubsection[\ L{\textquotesingle}\'emission
stimul\'ee]{\ L{\textquotesingle}\'emission stimul\'ee}
On a de nouveau lib\'eration d'\'energie sous la forme d'un photon
d'\'energie $E=E_{2}-E_{1}=h\cdot \nu $ \textsubscript{ }\`a cause de
la d\'esexcitation de l'atome. Mais contrairement \`a l'\'emission
spontan\'ee, c'est un autre photon qui provoque ce rayonnement. La
d\'esexcitation de l'atome est provoqu\'ee par l'arriv\'ee d'un photon
de m\^eme \'energie que celle qui serait lib\'er\'e par \'emission
spontan\'ee.

On obtient alors un rayonnement coh\'erent : le photon \'emis par
l{\textquotesingle}atome a la m\^eme longueur d{\textquotesingle}onde
et se propage dans la m\^eme direction que le photon incident. Le
rayonnement est amplifi\'e par l'\'emission stimul\'ee.

Il est int\'eressant de r\'ecup\'erer ce rayonnement, on cherche donc
\`a conna\^itre les conditions permettant d{\textquotesingle}avoir une
\'emission stimul\'ee au sein de la mati\`ere.

Soit

 $N_{1}$ le nombre d'atomes dans l'\'etat $E_{1}$ 

 $N_{2}$ le nombre d'atomes dans l'\'etat $E_{2}$ 

Pour avoir une \'emission stimul\'ee efficace, on doit avoir
$N_{2}>N_{1}$ . On obtient alors une inversion de population.

{\centering
\begin{minipage}{7.753cm}
{\itshape
[Warning: Draw object ignored]Illustration 3: Ph\'enom\`ene
d{\textquotesingle}\'emission stimul\'ee}
\end{minipage}
\par}

\subsection[\ Le LASER]{\bfseries \ Le LASER}
Le but du \textit{LASER} est d'entretenir cette inversion de population
et d'amplifier l'\'emission stimul\'ee afin de pouvoir utiliser le
rayonnement coh\'erent \'emis.

\subsubsection[\ Le pompage]{\ Le pompage}
Une excitation ext\'erieure maintient l'inversion de population. On
utilise le fait que certaines transitions entre les niveaux d'\'energie
soient non radiatives. Le cycle d\'ecrit par les atomes se d\'ecompose
en plusieurs \'etapes :

\liststyleLiv
\begin{enumerate}
\item On excite l'atome au niveau $E_{0}$ par pompage : on arrive au
niveau $E_{3}$ .
\item L'atome se d\'esexcite sans \'emettre de radiations pour arriver
au niveau $E_{2}$ . 
\item Lors du passage du niveau $E_{2}$ au niveau $E_{1}$ , on a une
\'emission laser.
\item A partir du niveau  $E_{1}$ , on a de nouveau une d\'esexcitation
non radiative jusqu'au niveau  $E_{0}$ .
\end{enumerate}
Si les transitions non radiatives sont plus rapides que l'\'emission
radiative, qui doit \^etre m\'etastable, les niveaux $E_{3}$ et $E_{1}$
se d\'epeuplent rapidement par ces transitions. On maintient
l'inversion de population gr\^ace au pompage, qui apporte
l{\textquotesingle}\'energie au syst\`eme.

Ce pompage peut s'effectuer au moyen de plusieurs dispositifs :

\liststyleLv
\begin{enumerate}
\item Le pompage optique ( lampe flash, autre laser )
\item Le pompage \'electrique
\item Le pompage chimique (par combustion)
\item Le pompage avec rayonnement ionisant (explosion nucl\'eaire)
\end{enumerate}
{\centering
\begin{minipage}{12.003cm}
{\itshape
[Warning: Draw object ignored]Illustration 4: Principe du pompage dans
un laser}
\end{minipage}
\par}

\subsubsection[\ L'amplification]{\ L'amplification}
On cherche maintenant \`a amplifier l'\'emission stimul\'ee afin
d{\textquotesingle}obtenir une onde \'electromagn\'etique de forte
amplitude en sortie.

Dans ce but, on va faire r\'esonner dans une cavit\'e le rayonnement
obtenu apr\`es le pompage. Le mat\'eriau actif est plac\'e entre deux
miroirs formant la cavit\'e. L'onde cr\'ee par pompage est amplifi\'ee
par des allers{}-retours en phase entre ces deux miroirs.

Pour favoriser l'\'emission stimul\'ee selon une direction pr\'ecise, on
choisit une cavit\'e de forme tr\`es allong\'ee dans la direction
d\'esir\'ee. Les rayonnements \'emis selon des directions diff\'erentes
ne seront alors pas amplifi\'es, mais le rayonnement \'emis dans
l{\textquotesingle}axe de la cavit\'e sera tr\`es fortement amplifi\'e.

Pour que les ondes \'electromagn\'etiques r\'efl\'echies par les miroirs
s{\textquotesingle}additionnent, elles doivent \^etre en phase. La
distance $L$ entre les miroirs ne peut donc avoir que des valeurs
pr\'ecises donn\'ees par :

{\centering
 $L=q\cdot {\frac{\lambda }{2}}$ 
\par}

\MakeUppercase{a}vec

 $q\in \mathbb{N}$ 

 $\lambda =\frac{c}{\nu }$ longueur d'onde


\bigskip

Il faut de plus que les miroirs soient parall\`eles afin de minimiser
les pertes lors du trajet de l'onde et de cr\'eer une addition des
amplitudes.

Cette cavit\'e est \'equivalente \`a un interf\'erom\`etre
Fabry{}-P\'erot.

\begin{minipage}{15.253cm}
{\itshape
[Warning: Draw object ignored]Illustration 5: Addition des ondes dans le
cas d{\textquotesingle}une longueur de cavit\'e adapt\'ee}
\end{minipage}

\begin{minipage}{15.253cm}
{\itshape
[Warning: Draw object ignored]Illustration 6: Destruction des ondes dans
le cas d{\textquotesingle}une longueur de cavit\'e inadapt\'ee}
\end{minipage}

{\bfseries
Le laser est donc constitu\'e de :}

\liststyleLvi
\begin{enumerate}
\item Un mat\'eriau actif : c'est de lui dont d\'epend la radiation
\'emise
\item Un dispositif de pompage : il doit pouvoir exciter les atomes
\item Une cavit\'e r\'esonante : elle permet
l{\textquotesingle}amplification de l{\textquotesingle}\'emission
stimul\'ee
\end{enumerate}
\subsection[\ Les modes de propagation]{\ Les modes de propagation}
\subsubsection[\ Les modes longitudinaux ]{\ Les modes longitudinaux }
On sait que la longueur de la cavit\'e doit \^etre multiple de $\lambda
/2$ . On aura donc g\'en\'eralement plusieurs longueurs
d{\textquotesingle}onde $\lambda $ possibles parmi le spectre
d'\'emission du mat\'eriau actif. On dit que le laser a plusieurs modes
longitudinaux. 

L'intervalle entre deux $\lambda $ correspondants \`a deux modes
longitudinaux adjacents est appel\'e intervalle spectral libre $\Delta
\nu $ . Il est caract\'eris\'e par :

{\centering
 $\Delta \nu =\frac{c\text{{\textquotesingle}}}{2\cdot
L}=\frac{c}{2\cdot n\cdot L}$ 
\par}

Avec

 $c\text{{\textquotesingle}}$ vitesse de la lumi\`ere dans le mat\'eriau

 $n$ indice optique du milieu


\bigskip

Si on cherche un laser monomode, c{\textquotesingle}est{}-\`a{}-dire
caract\'eris\'e par une seule longueur d'onde, on utilisera un
mat\'eriau dont la bande d'\'emission est la plus \'etroite possible.
Les longueurs d{\textquotesingle}onde $\lambda $ v\'erifiant $L=q\cdot
{\frac{\lambda }{2}}$ seront alors plus rares.

\subsubsection[\ Les modes transverses]{\ Les modes transverses}
Si les miroirs sont plac\'es \`a la bonne distance mais qu'ils ne sont
pas rigoureusement plans, \ il existera plusieurs zones de l'espace
pouvant devenir des cavit\'es r\'esonantes. Des modes de propagation
transverses appara\^itront. Au lieu d'un seul, plusieurs faisceaux se
dirigeant dans des directions diff\'erentes seront \'emis par le laser.

Il est difficile d'obtenir un laser monomode transverse car il faudrait
des miroirs parfaitement plans.

\subsection[\ Le laser \`a semi{}-conducteur]{\ Le laser \`a
\foreignlanguage{english}{semi}{}-conducteur}
\subsubsection[\ Les semi{}-conducteurs]{\ Les
\foreignlanguage{english}{semi}{}-conducteurs}
Les \foreignlanguage{english}{semi}{}-conducteurs ont des propri\'et\'es
de conduction du courant interm\'ediaire entre celles des isolants et
des conducteurs.

Dans un \foreignlanguage{english}{semi}{}-conducteur, le courant circule
gr\^ace :

\liststyleLvii
\begin{enumerate}
\item Aux \'electrons, qui sont libres de se d\'eplacer si on applique
un courant
\item Aux trous, qui sont des emplacements libres pour les \'electrons.
\end{enumerate}
On contr\^ole la densit\'e et le type des porteurs gr\^ace au dopage du
mat\'eriau (introduction d'impuret\'es).

\liststyleLviii
\begin{enumerate}
\item Un dopage de type P conduit \`a la production de trous. 
\item Un dopage de type N contient un exc\`es d'\'electrons.
\end{enumerate}
La jonction P{}-N est la juxtaposition d'une couche de mat\'eriau de
type N et d'une couche de mat\'eriau de type P. la zone active sera \`a
la jonction de ces deux couches.

On applique un potentiel n\'egatif sur le mat\'eriau N et un potentiel
positif sur le mat\'eriau P. On aura alors un exc\`es de porteurs de
part et d'autre de la jonction. Dans la jonction, il y a une
recombinaison entre \'electrons et trous : les \'electrons
{\guillemotleft}tombent{\guillemotright} dans les trous. Ils passent de
leur bande de conduction \`a la bande de valence : ils \'etaient dans
un \'etat libre et sont \`a pr\'esent li\'es \`a un atome. Ils passent
donc \`a un niveau d'\'energie plus bas. Ce changement de niveau
d{\textquotesingle}\'energie lib\`ere un photon dont l'\'energie est
environ \'egale \`a la diff\'erence entre le niveau li\'e et le niveau
libre de l'\'electron.

\begin{minipage}{15.253cm}
{\itshape
[Warning: Draw object ignored]Illustration 7: Processus
d{\textquotesingle}\'emission de photons dans une jonction P{}-N}
\end{minipage}

\subsubsection[\ Utilisation des semi{}-conducteurs dans un
laser]{\ Utilisation des \foreignlanguage{english}{semi}{}-conducteurs
dans un laser}
\foreignlanguage{french}{On peut donc r\'ealiser un laser \`a l'aide de
}\foreignlanguage{english}{semi}\foreignlanguage{french}{{}-conducteurs
en utilisant les photons qu'ils peuvent \'emettre. Le pompage
s'effectue par le passage d'un courant important dans la jonction. La
cavit\'e r\'esonante est constitu\'ee du mat\'eriau lui{}-m\^eme : les
faces de la jonction sont polies et donc
}\foreignlanguage{english}{semi}\foreignlanguage{french}{{}-r\'efl\'echissantes.
Le guidage dans la cavit\'e se fait par l'indice ou par le gain (Cette
deuxi\`eme m\'ethode n'est pas utilis\'ee dans les diodes du
laboratoire).}

{\selectlanguage{french}
\begin{minipage}{15.253cm}
{\itshape
[Warning: Draw object ignored]Illustration 8: Sch\'ema
d{\textquotesingle}un laser \`a semi conducteur}
\end{minipage}}

\section[\ Variation de la puissance optique \'emise en fonction de la
temp\'erature et de l{\textquotesingle}intensit\'e \'electrique
re\c{c}ue]{\ Variation de la puissance optique \'emise en fonction de
la temp\'erature et de l{\textquotesingle}intensit\'e \'electrique
re\c{c}ue}
\subsection[\ Objectifs]{\ Objectifs}
Le but de cette exp\'erience est de mesurer la variation de la puissance
optique \'emise par la diode n{\textdegree} en fonction de
l{\textquotesingle}intensit\'e \'electrique qu{\textquotesingle}elle
re\c{c}oit. Nous verrons comment cette caract\'eristique varie en
fonction de la temp\'erature de la diode. 

Si le mod\`ele pr\'esent\'e du fonctionnement du laser est correct, nous
devrions pouvoir distinguer deux r\'egimes : Le r\'egime
d{\textquotesingle}\'emission spontan\'ee si
l{\textquotesingle}intensit\'e \'electrique est insuffisante pour
cr\'eer l{\textquotesingle}inversion de population, et le r\'egime
d{\textquotesingle}\'emission stimul\'ee, si
l{\textquotesingle}inversion de population est effectu\'ee.

Nous essaierons ensuite de visualiser certaines caract\'eristiques de la
diode, dont son intensit\'e de seuil, et son rendement diff\'erentiel
externe.

\subsection[\ Manipulation]{\ Manipulation}
La manipulation consiste dans un premier temps \`a fixer la
temp\'erature de la diode. Un courant d{\textquotesingle}intensit\'e
sp\'ecifi\'ee est ensuite inject\'e dans la diode, dont on \'evalue
alors la puissance optique de sortie \`a l{\textquotesingle}aide de la
photodiode int\'egr\'ee dans le bo\^itier.

L{\textquotesingle}interface de contr\^ole de
l{\textquotesingle}ordinateur envoie une consigne
d{\textquotesingle}intensit\'e \`a un bo\^itier \textit{Spectra Diode
Labs}\textup{. Le bo\^itier} transmet l{\textquotesingle}intensit\'e
\'electrique vers la diode laser. Le bo\^itier \textit{Spectra Diode
Labs} permet de r\'egler manuellement la temp\'erature appliqu\'ee sur
la diode \`a l{\textquotesingle}aide d{\textquotesingle}un module \`a
effet Pelletier.

\begin{minipage}{15.253cm}
{\itshape
[Warning: Draw object ignored]Illustration 9: Sch\'ema de principe de la
premi\`ere exp\'erience}
\end{minipage}

Une photodiode plac\'ee sur la face arri\`ere de la diode laser permet
renvoie une tension image de la puissance optique, qui est envoy\'ee
\`a l{\textquotesingle}interface de l{\textquotesingle}ordinateur via
un voltm\`etre.

\begin{minipage}{15.253cm}
{\itshape
[Warning: Draw object ignored]Illustration 10: Contenu des bo\^itiers
diode laser du laboratoire}
\end{minipage}

\subsection[\ R\'esultats]{\ R\'esultats}
\subsubsection[\ R\'esultats bruts]{\ R\'esultats bruts}
Pour une temp\'erature de 25{\textdegree}C, les r\'esultats obtenus sont
pr\'esent\'es sur l{\textquotesingle}Illustration 11.

\begin{minipage}{15.334cm}
{\itshape
 [Warning: Image not found] Illustration
11\label{seq:refIllustration10}: Variation de la puissance optique de
sortie en fonction de l{\textquotesingle}intensit\'e pour une
temp\'erature de 25{\textdegree}C}
\end{minipage}

Les r\'esultats bruts obtenus sont pr\'esent\'es graphiquement sur
l{\textquotesingle}Illustration 12 pour des temp\'eratures comprises
entre {}-10{\textdegree}C et 30{\textdegree}C .

\begin{minipage}{15.334cm}
{\itshape
 [Warning: Image not found] Illustration
12\label{seq:refIllustration11}: Variation de la puissance optique de
sortie en fonction de l{\textquotesingle}intensit\'e pour des
temp\'eratures variant entre {}-10{\textdegree}C et 30{\textdegree}C}
\end{minipage}

On remarque quelques imperfections sur les courbes. Ces imperfections
peuvent \^etre dues \`a des difficult\'es dans
l{\textquotesingle}asservissement de la temp\'erature de la diode, mais
aussi \`a des changements de modes de fonctionnement du laser (voir)

\subsubsection[\ Mod\'elisation]{\ Mod\'elisation}
Il est possible de mod\'eliser chaque courbe en distinguant 2 segments
correspondant aux \ 2 r\'egimes d{\textquotesingle}\'emission. On
pourra ainsi \'evaluer l{\textquotesingle}intensit\'e de seuil, qui
sera l{\textquotesingle}abscisse du point de rupture de la pente, et le
rendement diff\'erentiel externe, qui sera le coefficient directeur du
second segment.

Sur l{\textquotesingle}Illustration 13, on remarque que la
mod\'elisation est tr\`es fid\`ele \`a la r\'ealit\'e.

{\mdseries\upshape
\begin{minipage}{15.334cm}
{\itshape
 [Warning: Image not found] Illustration
13\label{seq:refIllustration12}: Variation de la puissance optique de
sortie en fonction de l{\textquotesingle}intensit\'e : comparaison
entre la mod\'elisation et la mesure}
\end{minipage}}

Cependant, l{\textquotesingle}Illustration 14 montre
qu{\textquotesingle}il existe une zone de 5 mA autour de
l{\textquotesingle}intensit\'e de seuil sur laquelle
l{\textquotesingle}approximation en 2 segments ne semble plus valable.

\begin{minipage}{15.334cm}
{\itshape
 [Warning: Image not found] Illustration
14\label{seq:refIllustration13}: Variation de la puissance optique de
sortie en fonction de l{\textquotesingle}intensit\'e : comparaison
entre la mod\'elisation et la mesure autour du courant de seuil}
\end{minipage}

La diff\'erence entre la mod\'elisation et la r\'ealit\'e peut \^etre
due \`a plusieurs effets physiques, mais aussi \`a des erreurs de
mesure. En effet, la r\'egulation de temp\'erature peut conna\^itre des
difficult\'es si l{\textquotesingle}intensit\'e inject\'ee varie trop
rapidement (Cas d{\textquotesingle}un pas de mesure trop \'elev\'e).

\subsubsection[\ Influence de la temp\'erature]{\ Influence de la
temp\'erature}
Gr\^ace aux donn\'ees issues de l{\textquotesingle}exp\'erience , nous
remarquons que la temp\'erature influe sur plusieurs grandeurs
caract\'eristiques de la diode laser. Ainsi,
l{\textquotesingle}Illustration 15 montre que
l{\textquotesingle}intensit\'e de seuil augmente avec la temp\'erature
de mani\`ere non n\'egligeable.

\begin{minipage}{15.334cm}
{\itshape
 [Warning: Image not found] Illustration
15\label{seq:refIllustration14}: Variation du courant de seuil en
fonction de la temp\'erature}
\end{minipage}

L{\textquotesingle}Illustration 16 montre de plus que le rendement
diff\'erentiel externe d\'ecro\^it consid\'erablement lorsque la
temp\'erature augmente.

\begin{minipage}{15.334cm}
{\itshape
 [Warning: Image not found] Illustration
16\label{seq:refIllustration15}: Variation du rendement diff\'erentiel
externe en fonction de la temp\'erature}
\end{minipage}

Ces variations des performances de la diode avec la temp\'erature
peuvent avoir des effets n\'efastes dans certaines applications. Ainsi
un syst\`eme utilisant une diode laser pour moduler un signal pourrait
devenir inop\'erant si la temp\'erature monte trop (Selon les saisons,
l{\textquotesingle}heure du jour etc...). Il faut donc tenir compte des
effets de la temp\'erature dans l{\textquotesingle}\'elaboration
d{\textquotesingle}un syst\`eme bas\'e sur une diode laser.
L{\textquotesingle}Illustration 17 nous montre par exemple
l{\textquotesingle}\'evolution de la puissance optique \'emise par la
diode en fonction de la temp\'erature pour une intensit\'e \'electrique
d{\textquotesingle}entr\'ee fix\'ee \`a 300mA.

\begin{minipage}{15.334cm}
{\itshape
 [Warning: Image not found] Illustration
17\label{seq:refIllustration16}: Variation de la puissance optique de
sortie en fonction de la temp\'erature pour une intensit\'e
\'electrique de 300mA}
\end{minipage}

\section[\ \'Etude des modes de propagation transverses]{\ \'Etude des
modes de propagation transverses}
\subsection[\ Objectifs]{\ Objectifs}
Le but de cette exp\'erience est de caract\'eriser les modes transverses
de la diode n{\textdegree}. Nous voulons conna\^itre le diam\`etre et
la hauteur du faisceau obtenu \`a la sortie de la diode. Si la diode
n'est pas monomode longitudinal, nous devrions trouver plusieurs
faisceaux \'emergent. Nous chercherons alors \`a d\'eterminer leur
distribution spatiale.

\subsection[\ Manipulation]{\ Manipulation}
La manipulation consiste \`a faire pivoter la diode autour
d{\textquotesingle}un axe. La photodiode est plac\'ee \`a une distance
donn\'ee de la diode laser, et on mesure l{\textquotesingle}intensit\'e
lumineuse re\c{c}ue par la photodiode en fonction de
l{\textquotesingle}angle de rotation de la diode autour de son axe.

La rotation de la diode est motoris\'ee. L{\textquotesingle}interface de
contr\^ole de l{\textquotesingle}ordinateur envoie une consigne de
d\'eplacement angulaire. On obtient alors, en sortie de la photodiode,
une image de la r\'epartition spatiale de la puissance optique \'emise.

Le balayage s'effectue en 2 fois : une premi\`ere fois parall\`element
\`a la jonction, une seconde fois perpendiculairement. Le courant
inject\'e dans le laser et la temp\'erature ne varient pas lors de
cette manipulation.

\begin{minipage}{15.253cm}
{\itshape
[Warning: Draw object ignored]Illustration 18: Sch\'ema de principe de
la seconde exp\'erience}
\end{minipage}

\begin{minipage}{15.253cm}
{\itshape
[Warning: Draw object ignored]Illustration 19: Caract\'erisation de la
puissance optique relative en fonction de l{\textquotesingle}angle par
rapport \`a l{\textquotesingle}axe de la diode}
\end{minipage}

\subsection[\ R\'esultats]{\ R\'esultats}
\subsubsection[\ Caract\'erisation spatiale]{\ Caract\'erisation
spatiale}
L{\textquotesingle}Illustration 20 et l{\textquotesingle}Illustration 21
nous montrent les allures des profils d{\textquotesingle}intensit\'e
lumineuse dans les plans perpendiculaire \`a la jonction parall\`ele
\`a la jonction. Nous remarquons que la tache est beaucoup plus
\'etal\'ee dans la direction perpendiculaire \`a la jonction. Ce
r\'esultat \'etait pr\'evisible, car la jonction ayant une \'epaisseur
tr\`es faible et une largeur plus importante, la diffraction des ondes
est plus forte dans le sens perpendiculaire \`a la jonction,
d{\textquotesingle}o\`u l{\textquotesingle}allure des t\^aches.

\begin{minipage}{15.334cm}
{\itshape
 [Warning: Image not found] Illustration
20\label{seq:refIllustration19}: Variation de la puissance optique
relative en fonction de l{\textquotesingle}angle dans le plan
parall\`ele \`a la jonction}
\end{minipage}

\begin{minipage}{15.334cm}
{\itshape
 [Warning: Image not found] Illustration
21\label{seq:refIllustration20}: Variation de la puissance optique
relative en fonction de l{\textquotesingle}angle dans le plan
perpendiculaire \`a la jonction}
\end{minipage}

\subsubsection[\ Influence de l{\textquotesingle}orientation de la diode
laser dans les mesures]{\ Influence de l{\textquotesingle}orientation
de la diode laser dans les mesures}
L{\textquotesingle}Illustration 22 montre le profil de la puissance
lumineuse en fonction de l{\textquotesingle}angle dans le plan
perpendiculaire \`a la jonction, en orientant d{\textquotesingle}abord
la diode dans un sens par rapport au r\'ef\'erentiel du laboratoire,
puis en la retournant de 180{\textdegree}. On remarque que les
accidents de la courbe, notamment le petit pic situ\'e vers
{}-20{\textdegree}, sont pr\'esents au m\^eme endroit sur les deux
courbes. La largeur \`a mi{}-hauteur est elle aussi inchang\'ee dans
les deux mesures. Les allures des courbes sont donc intrins\`eques \`a
la diode, et ne proviennent pas d{\textquotesingle}une quelconque
perturbation due aux conditions d{\textquotesingle}exp\'erimentation.

\begin{minipage}{15.334cm}
{\itshape
 [Warning: Image not found] Illustration
22\label{seq:refIllustration21}: Influence de
l{\textquotesingle}orientation de la diode pour les mesures dans le
plan perpendiculaire \`a la jonction}
\end{minipage}

\subsubsection[\ Influence de la distance entre la diode laser et la
photodiode]{\ Influence de la distance entre la diode laser et la
photodiode}
L{\textquotesingle}Illustration 23 repr\'esente deux mesures du profil
de la puissance lumineuse en fonction de l{\textquotesingle}angle dans
le plan perpendiculaire \`a la jonction, pour deux distances
diff\'erentes entre la diode laser et la photodiode. On remarque que
les deux profils sont semblables, cependant des ph\'enom\`enes de
propagation longitudinale pourraient les rendre dissemblables. 

\begin{minipage}{15.334cm}
{\itshape
 [Warning: Image not found] Illustration
23\label{seq:refIllustration22}: Influence de la distance entre la
diode laser et la photodiode pour les mesures dans le plan
perpendiculaire \`a la jonction}
\end{minipage}

On remarque sur l{\textquotesingle}Illustration 24, que lorsque la
photodiode est proche de la diode laser, un ph\'enom\`ene
d{\textquotesingle}interf\'erence semble appara\^itre lorsque la
photodiode est dans l{\textquotesingle}axe de la diode laser. Ces raies
peuvent \^etre caus\'ees par des interf\'erences dues aux r\'eflexions
du faisceau laser sur la photodiode et sur la diode laser (voir)

\begin{minipage}{15.334cm}
{\itshape
 [Warning: Image not found] Illustration
24\label{seq:refIllustration23}: Apparition d{\textquotesingle}un
ph\'enom\`ene d{\textquotesingle}interf\'erence pour des angles petits
et une faible distance entre la diode laser et la photodiode}
\end{minipage}

\begin{minipage}{15.253cm}
{\itshape
[Warning: Draw object ignored]Illustration 25: Sch\'ema de principe
illustrant une cause possible du ph\'enom\`ene observ\'e}
\end{minipage}

\subsubsection[\ Diff\'erents modes de propagation
transverse]{\ Diff\'erents modes de propagation transverse}
Lors d{\textquotesingle}une premi\`ere mesure du profil de la puissance
lumineuse en fonction de l{\textquotesingle}angle dans le plan
parall\`ele \`a la jonction, repr\'esent\'e sur
l{\textquotesingle}Illustration 20, nous remarquons un pic principal
tr\`es visible autour de 0{\textdegree}, et un second pic plus faible
autour de {}-6{\textdegree}. Nous pouvons donc supposer
l{\textquotesingle}existence de plusieurs modes de propagation
transverse, correspondant chacun \`a un pic.

L{\textquotesingle}Illustration 26 apporte une confirmation \`a notre
hypoth\`ese. En effet, sur les profils repr\'esent\'es pour
diff\'erentes valeurs de l{\textquotesingle}intensit\'e \'electrique
re\c{c}ue par la diode montrent clairement qu{\textquotesingle}il
existe 2 modes principaux :

\liststyleLix
\begin{enumerate}
\item En dessous de 300mA, le pic principal est situ\'e autour de
{}-6{\textdegree}, accompagn\'e par un autre pic autour de
{}-8{\textdegree}.
\item Au del\`a d{\textquotesingle}une valeur critique de
l{\textquotesingle}intensit\'e, on remarque que les deux pics
pr\'ec\'edents disparaissent compl\`etement, et il appara\^it un pic
tr\`es marqu\'e autour de 0{\textdegree}.
\end{enumerate}
\begin{minipage}{15.334cm}
{\itshape
 [Warning: Image not found] Illustration
26\label{seq:refIllustration25}: Variation de la puissance optique
relative en fonction de l{\textquotesingle}angle dans le plan parallele
\`a la jonction pour deux modes de propagation transverse diff\'erents}
\end{minipage}

Notre premi\`ere mesure a donc \'et\'e effectu\'ee pour une valeur de
l{\textquotesingle}intensit\'e proche de la valeur critique, puisque
les deux mode sont visibles.

Il est \`a remarquer que le passage d{\textquotesingle}un mode \`a
l{\textquotesingle}autre est sujet \`a un hysteresis,
puisqu{\textquotesingle}il n{\textquotesingle}intervient pas pour la
m\^eme valeur de l{\textquotesingle}intensit\'e si
l{\textquotesingle}on va dans le sens de la diminution de
l{\textquotesingle}intensit\'e ou si l{\textquotesingle}on va dans le
sens de l{\textquotesingle}augmentation de
l{\textquotesingle}intensit\'e.

De plus, dans la zone situ\'ee autour de l{\textquotesingle}intensit\'e
critique, le signal n{\textquotesingle}est pas stable dans le temps. En
effet, le laser peut alors basculer d{\textquotesingle}un mode \`a
l{\textquotesingle}autre sans manipulation ext\'erieure.

\section[\ \'Etude des modes de propagation longitudinaux]{\ \'Etude des
modes de propagation longitudinaux}
\subsection[\ Objectifs]{\ Objectifs}
Le but de cette exp\'erience est de caract\'eriser les modes de
propagation longitudinaux de la diode. Nous voulons conna\^itre les
diff\'erentes longueurs d'ondes qui ont \'et\'e amplifi\'ees dans la
cavit\'e r\'esonante et qui composent le rayonnement \'emis. Nous
cherchons \'egalement \`a d\'eterminer les puissances associ\'ees \`a
chaque longueur d'onde.

\subsection[\ Manipulation]{\ Manipulation}
Un monochromateur est utilis\'e dans cette manipulation. Il permet de
s\'eparer les diff\'erentes longueurs d'onde du faisceau.

Le faisceau de la diode est amen\'e en entr\'ee du monochromateur.
Gr\^ace au r\'eseau, on r\'ealise la diffraction de ce faisceau :
l'angle de r\'efraction d\'epend de la longueur d'onde. Les
diff\'erents rayons correspondant aux diff\'erentes longueurs d'ondes
ont une direction diff\'erente apr\`es r\'eflexion sur le r\'eseau. En
mesurant cet angle de d\'eviation, on aura la longueur d'onde.

\begin{minipage}{15.253cm}
{\itshape
[Warning: Draw object ignored]Illustration 27: Sch\'ema de principe de
la troisi\`eme exp\'erience}
\end{minipage}

On fait pivoter le r\'eseau pour d\'eterminer l{\textquotesingle}angle
de d\'eviation. Dans notre manipulation, la commande de d\'eplacement
angulaire est donn\'ee par l'ordinateur et ce d\'eplacement est
motoris\'e.

Une photodiode plac\'ee en sortie du monochromateur donne une image de
la puissance optique associ\'ee \`a chaque angle du r\'eseau de
diffraction, et donc \`a chaque longueur d'onde.

\begin{minipage}{15.253cm}
{\itshape
[Warning: Draw object ignored]Illustration 28: Principe de
fonctionnement de la spectroscopie utilisant un monochromateur}
\end{minipage}

\subsection[\ R\'esultats]{\ R\'esultats}
\subsubsection[\ Mod\'elisation]{\ Mod\'elisation}
LIllustration 29{\textquotesingle} pr\'esente l{\textquotesingle}allure
th\'eorique des raies. Les raies sont espac\'ees de
l{\textquotesingle}intervalle spectral libre (voir), et elles sont
modul\'ees par le spectre d{\textquotesingle}\'emission du mat\'eriau
actif, qui a une forme caract\'eristique due \`a la r\'epartition
gaussienne des vitesses des atomes dans le milieu actif. Le produit de
ces deux fonctions donne une allure th\'eorique constitu\'ee de
plusieurs raies dont les plus intenses sont au centre.

\begin{minipage}{15.334cm}
{\itshape
 [Warning: Image not found] Illustration
29\label{seq:refIllustration28}: Allure th\'eorique du spectre de la
lumi\`ere \'emise par un laser multimodes}
\end{minipage}

Sur l{\textquotesingle}Illustration 30, nous remarquons que la
mod\'elisation est assez fid\`ele \`a la r\'ealit\'e, mais que
l{\textquotesingle}enveloppe des courbes r\'eelles ne correspond pas
exactement \`a la courbe gaussienne th\'eorique.

\begin{minipage}{15.334cm}
{\itshape
 [Warning: Image not found] Illustration
30\label{seq:refIllustration29}: Comparaison des allures th\'eorique et
r\'eelles du spectre de la lumi\`ere \'emise par un laser multimodes}
\end{minipage}

\subsubsection[\ Influence de la temp\'erature]{\ Influence de la
temp\'erature}
L{\textquotesingle}Illustration 31 nous montre la superposition des
diff\'erents spectres pour des temp\'eratures variant entre
{}-10{\textdegree}C et 30{\textdegree}C.

\begin{minipage}{15.334cm}
{\itshape
 [Warning: Image not found] Illustration
31\label{seq:refIllustration30}: Superposition des diff\'erents
spectres correspondant pour diff\'erentes valeurs de la temp\'erature}
\end{minipage}

L{\textquotesingle}Illustration 32 nous montre la variation de la
longueur d{\textquotesingle}onde centrale des spectres en fonction de
la temp\'erature. Nous remarquons que l{\textquotesingle}augmentation
de la temp\'erature conduit \`a une augmentation de la longueur
d{\textquotesingle}onde de la lumi\`ere \'emise par le laser.

\begin{minipage}{15.334cm}
{\itshape
 [Warning: Image not found] Illustration
32\label{seq:refIllustration31}: Variation de la longueur
d{\textquotesingle}onde centrale en fonction de la temp\'erature}
\end{minipage}

\subsubsection[\ Influence de la puissance optique]{\ Influence de la
puissance optique}
L{\textquotesingle}Illustration 33 nous montre la superposition des
diff\'erents spectres pour des puissances optiques \ variant entre 0mW
et 200mW.

\begin{minipage}{15.334cm}
{\itshape
 [Warning: Image not found] Illustration
33\label{seq:refIllustration32}: Superposition des diff\'erents
spectres correspondant pour diff\'erentes valeurs de la puissance}
\end{minipage}

L{\textquotesingle}Illustration 34 nous montre la variation de la
longueur d{\textquotesingle}onde centrale des spectres en fonction de
la puissance optique. Nous remarquons que
l{\textquotesingle}augmentation de la puissance conduit \`a une
augmentation de la longueur d{\textquotesingle}onde de la lumi\`ere
\'emise par le laser. 

\begin{minipage}{15.334cm}
{\itshape
 [Warning: Image not found] Illustration
34\label{seq:refIllustration33}: Variation de la longueur
d{\textquotesingle}onde centrale en fonction de la puissance optique}
\end{minipage}

\section[\ Conclusion]{\ Conclusion}
\section[\ Annexes]{\ Annexes}
\subsection[\ Fiche technique de la diode]{\ Fiche technique de la
diode}
\clearpage{\sffamily\bfseries
Bibliographie}

Opto\'electronique Composants photoniques et fibres optiques, Zeno
Toffano,

Laser guide \foreignlanguage{english}{book} (2nd
\foreignlanguage{english}{edition}), \foreignlanguage{english}{Jeff}
Hecht,

Introduction \`a la physique du laser, \foreignlanguage{english}{Bela}
A. Lengyel,


\bigskip

[Warning: List of illustrations ignored]


\bigskip
\end{document}
