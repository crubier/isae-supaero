% -*- mode: latex; tex-main-file: "cantor.tex" -*-
% $Id: pages-liminaires.tex,v 1.31 2005/03/11 09:51:09 chope Exp $
\TitreNote{Manuel d'utilisation de la biblioth�que CANTOR}
\ProjetNote{CANTOR}
\NumEd{5}
\DateEd{04/09/2000}
\NumRev{11}
\DateRev{04/03/2005}
\RefNote{ESPACE/MS/CHOPE/CANTOR/MU/001}
\RedigePar{\begin{tabular}[t]{@{}l@{}}
                L. Maisonobe, O. Queyrut\\G. Prat, F. Auguie, S. Vresk
           \end{tabular}}{CS SI/ESPACE/FDS}
\ValidePar{G. Prat}{CS SI/ESPACE/FDS}
\PourAppli{C. Fernandez-Martin}{CS SI/ESPACE/FDS}

\Hierarchie{
\textsc{Communications et Syst�mes}\vspace{3mm}\\
\textsc{Syst�mes d'Informations} \vspace{2ex} \\
\textsc{Direction Espace}
}


\ResumeNote{Ce manuel d'utilisation provient d'un document CNES r�dig�
par Luc Maisonobe. Il d�crit la biblioth�que des Composants
d'Analyse Num�rique Traduits sous forme d'Objets R�utilisables
\textsc{cantor}. Cette biblioth�que fournit des classes \langage{c++}
implantant des composants g�om�triques de base comme les vecteurs et
rotations en dimension 3 et les �l�ments simples sur la sph�re unit�,
des classes d'analyse comme les d�riv�es automatiques � l'ordre un ou
deux, les tableaux de variations et des algorithmes de r�solution de
z�ros, ainsi que quelques composants classiques commes les polyn�mes,
les m�thodes de moindres carr�s lin�aires et les approximations
fonctionnelles.}

\MotsClefsNote{\shortstack[l]{biblioth�que, math�matiques, \langage{c++}}}

\GestionConf{Oui}
\RespGestionConf{\textsc{Luc Maisonobe - CSSI}}

\Modifications{4}{0}{03/08/1999}{SCS/CANTOR/MU/99-001, passage de la
documentation en version CS}
\Modifications{4}{1}{12/08/1999}{SCS/CANTOR/MU/99-001, modification de
la taille du tableau des m�thodes publiques de Arc et d'Intervalle}
\Modifications{4}{2}{12/07/2000}{SCS/CANTOR/MU/99-001, ajout de la
section concernant les �volutions entre les versions}
\Modifications{5}{0}{04/09/2000}{SCS/CANTOR/MU/2000-001, passage en
feuille de style notechope}
\Modifications{5}{1}{22/11/00}{SCS/CANTOR/MU/2000-001, description des
modifications de configuration entre les versions 5.3 et 5.4}
\Modifications{5}{2}{05/12/00}{SCS/CANTOR/MU/2000-001, description des
modifications de configuration entre les versions 5.4 et 5.5, ajout de
la description de \texttt{cantor-config}}
\Modifications{5}{3}{16/02/01}{SCS/CANTOR/MU/2000-001, description des
modifications de configuration entre les versions 5.5 et 5.6}
\Modifications{5}{4}{22/06/01}{SCS/CANTOR/MU/2000-001, description des
modifications de configuration entre les versions 5.6 et 6.0}
\Modifications{5}{5}{23/08/01}{SCS/CANTOR/MU/2000-001, description des
corrections entre les versions 6.0 et 6.1}
\Modifications{5}{6}{23/08/01}{SCS/CANTOR/MU/2000-001, description des
corrections entre les versions 6.1 et 6.2}
\Modifications{5}{7}{31/01/02}{ESPACE/MS/CHOPE/CANTOR/MU/001,
description des corrections entre les versions 6.2 et 6.3}
\Modifications{5}{8}{09/09/02}{ESPACE/MS/CHOPE/CANTOR/MU/001,
description des corrections entre les versions 6.3 et 7.0}
\Modifications{5}{9}{28/03/03}{ESPACE/MS/CHOPE/CANTOR/MU/001,
description des corrections entre les version 7.0 et 7.1 et ajout des
descriptions des constructeurs et destructeurs ajout�s dans la version 7.1}
\Modifications{5}{10}{28/07/03}{ESPACE/MS/CHOPE/CANTOR/MU/001,
description des �volutions introduites en version 7.2}
\Modifications{5}{11}{04/03/05}{ESPACE/MS/CHOPE/CANTOR/MU/001,
description des �volutions introduites en version 7.3}

\CausesEvolution{description des �volutions introduites en version 7.3}

\DiffusionInterne[2 exemplaires]{CS SI}{ESPACE/FDS}{}

\DiffusionExterne[3 exemplaires]{CNES}{DCT/SB/MS}{}
