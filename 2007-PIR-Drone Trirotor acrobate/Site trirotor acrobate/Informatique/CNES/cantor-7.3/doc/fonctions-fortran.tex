% -*- mode: latex; tex-main-file: "cantor.tex" -*-
% $Id: fonctions-fortran.tex,v 1.2 2001/06/05 16:12:46 cantor Exp $
\subsection{sous-routines \langage{fortran}}
\label{sec:sous-routines-fortran}
Seules les fonctions d'utilisation des rotations en dimension 3 sont
disponibles � partir du langage \langage{fortran}. Ces fonctions
n�cessitent le passage d'un ou de plusieurs tableaux de quatre r�els
pour contenir les composantes du quaternion repr�sentant la
rotation. Ces composantes sont rarement utiles pour l'appelant, il se
contente g�n�ralement de passer les tableaux � la biblioth�que
\bibliotheque{cantor}, laquelle les initialise dans le cas de
fonctions construisant des rotations, et les utilise dans les autres
cas.

\begin{tableFonctionsFixe}{fonctions FORTRAN}
{\label{tab:fonctions-fortran}}
{construit la rotation \argument{q} r�sultant de la composition
\argument{q2} $\circ$ \argument{q1}}

\signature{integer function \fonc{RotAxeAngle} (\argument{q},
\argument{axe}, \argument{angle},
\argument{message})}{\hspace{\largeurPar}double precision \argument{q}
(4)\\double precision \argument{axe} (3), \argument{angle}\\character*(*)
\argument{message}} & construit la rotation \argument{q} � partir de
l'\argument{axe} et de l'\argument{angle}, retourne un code non nul en
cas de probl�me, et initialise le \argument{message} d'erreur\\

\signature{integer function
\fonc{RotU1U2V1V2}}{\hspace{-\largeurPar}>\hspace{1cm}(\argument{q},
\argument{u1}, \argument{u2}, \argument{v1}, \argument{v2},
\argument{message})\\double precision \argument{q} (4)\\double
precision \argument{u1} (3), \argument{u2} (3)\\double precision
\argument{v1} (3), \argument{v2} (3)\\character*(*)
\argument{message}} & construit la rotation \argument{q} � partir des
deux vecteurs \argument{u1} et \argument{u2} et de leurs images
\argument{v1} et \argument{v2}, retourne un code non nul en cas de
probl�me, et initialise le \argument{message} d'erreur\\

\signature{integer function \fonc{RotU1V1} (\argument{q},
\argument{u1}, \argument{v1},
\argument{message})}{\hspace{\largeurPar}double precision \argument{q}
(4)\\double precision \argument{u1} (3),
\argument{v1} (3)\\character*(*) \argument{message}} & construit la
rotation \argument{q} � partir du vecteur \argument{u1} et de son
image \argument{v1} (une solution est choisie arbitrairement dans
l'infinit� possible), retourne un code non nul en cas de probl�me, et
initialise le \argument{message} d'erreur\\

\signature{integer function \fonc{RotMatrice} (\argument{q},
\argument{m}, \argument{seuil},
\argument{message})}{\hspace{\largeurPar}double precision \argument{q}
(4)\\double precision \argument{m} (3, 3),
\argument{seuil}\\character*(*) \argument{message}} & construit la
rotation \argument{q} � partir de la matrice \argument{m} en
corrigeant �ventuellement sa non-orthogonalit� d'au plus
\argument{seuil} (au sens de la norme de \textsc{Frobenius}), retourne
un code non nul en cas de probl�me, et initialise le
\argument{message} d'erreur\\

\signature{integer function \fonc{RotTroisAngles}}{\hspace{-\largeurPar}>\hspace{1cm}(\argument{q},
\argument{ordre}, \argument{alpha1}, \argument{alpha2},
\argument{alpha3}, \argument{message})\\double
precision \argument{q} (4)\\ double precision \argument{alpha1},
\argument{alpha2}, \argument{alpha3}\\ character*(*)
\argument{message}} & construit la rotation \argument{q} � partir de
trois rotations �l�mentaires dans l'\argument{ordre} sp�cifi� (qui
doit �tre conforme � l'un des \texttt{PARAMETER} d�clar�s dans le
fichier \texttt{cantor/cantdefs.f} que l'on peut inclure par
\texttt{\#include} ou par \texttt{INCLUDE} au niveau du fichier
appelant), retourne un code non nul en cas de probl�me, et initialise
le \argument{message} d'erreur\\

\signature{subroutine \fonc{RotInverse} (\argument{q},
\argument{qInitiale})}{\hspace{\largeurPar}double precision
\argument{q} (4), \argument{qInitiale} (4)} &
construit la rotation \argument{q} inverse de \argument{qInitiale}, en
se contentant d'inverser un �l�ment du quaternion (cette op�ration est
donc tr�s peu co�teuse en temps de calcul)\\

\signature{subroutine \fonc{RotComposee} (\argument{q}, \argument{q1},
\argument{q2})}{\hspace{\largeurPar}double precision \argument{q}
(4), \argument{q1} (4), \argument{q2} (4)} &
construit la rotation \argument{q} r�sultant de la composition
\argument{q2} $\circ$ \argument{q1}\\

\hline

\signature{subroutine \fonc{AxeRot} (\argument{q},
\argument{axe})}{\hspace{\largeurPar}double precision \argument{q}
(4), \argument{axe} (3)}& extrait l'\argument{axe} de la rotation
\argument{q}\\

\signature{subroutine \fonc{AngleRot} (\argument{q},
\argument{angle})}{\hspace{\largeurPar}double precision \argument{q}
(4), \argument{angle}} & extrait l'\argument{angle} de la rotation
\argument{q}\\

\signature{subroutine \fonc{AxeAngleRot} (\argument{q},
\argument{axe}, \argument{angle})}{\hspace{\largeurPar}double
precision \argument{q} (4), \argument{axe} (3), \argument{angle}} &
extrait l'\argument{axe} et l'\argument{angle} de la rotation
\argument{q}\\

\signature{subroutine \fonc{MatriceRot} (\argument{q},
\argument{m})}{\hspace{\largeurPar}double precision \argument{q} (4),
\argument{m} (3, 3)} & extrait la matrice \argument{m} de la rotation
\argument{q}\\

\signature{integer function
\fonc{TroisAnglesRot}}{\hspace{-\largeurPar}>\hspace{1cm}(\argument{q},
\argument{ordre}, \argument{alpha1}, \argument{alpha2},
\argument{alpha3}, \argument{message})\\double precision \argument{q}
(4)\\ double precision \argument{alpha1}, \argument{alpha2},
\argument{alpha3}\\ character*(*) \argument{message}} & extrait de la
rotation \argument{q} les angles des rotations �l�mentaires dans
l'\argument{ordre} sp�cifi� (qui doit �tre conforme � l'un des
\texttt{PARAMETER} d�clar�s dans le fichier \texttt{cantor/cantdefs.f}
que l'on peut inclure par \texttt{\#include} ou par \texttt{INCLUDE}
au niveau du fichier appelant), retourne un code non nul en cas de
probl�me, et initialise le \argument{message} d'erreur\\

\hline

\signature{subroutine \fonc{AppliqueRot} (\argument{q}, \argument{u},
\argument{uPrime})}{\hspace{\largeurPar}double precision \argument{q}
(4)\\ double precision \argument{u} (3), \argument{uPrime} (3)} &
applique la rotation \argument{q} au vecteur \argument{u} et met
l'image dans le tableau \argument{uPrime}\\

\signature{subroutine \fonc{AppliqueRotInverse} (\argument{q},
\argument{uPrime}, \argument{u})}{\hspace{\largeurPar}double precision
\argument{q} (4)\\double precision \argument{uPrime} (3), \argument{u}
(3)} & applique l'inverse de la rotation \argument{q} au
\argument{uPrime} et met l'image r�ciproque dans le tableau
\argument{u}\\

\end{tableFonctionsFixe}
