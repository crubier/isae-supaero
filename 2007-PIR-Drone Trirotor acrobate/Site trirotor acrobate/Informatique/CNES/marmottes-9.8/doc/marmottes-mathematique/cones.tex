% -*- mode: latex; tex-main-file: "marmottes-mathematique.tex" -*-
% $Id: cones.tex,v 1.7 2001/04/05 10:11:50 luc Exp $
\cleardoublepage\section{Intersection de c�nes sur la sph�re}
\label{sec:intersection}
Les champs de vue des capteurs g�om�triques sont d�finis comme des
combinaisons plus ou moins complexes de c�nes (r�unions et
intersections imbriqu�es). De plus, dans le cas des capteurs
de limbe, le bord de l'astre observ� est �galement vu par le capteur
comme un c�ne. Enfin, les consignes g�om�triques sont mod�lis�es par
des c�nes.

On est donc souvent amen� � rencontrer des intersections de c�nes sur
la sph�re unit� centr�e sur le satellite.

Soient deux c�nes ($\vec{u}$,~$\alpha$) et ($\vec{v}$,~$\beta$) non
coaxiaux (voir figure~\ref{fig:intersection}).
\begin{figure}[htbp]\caption{\label{fig:intersection}intersection de c�nes}
\begin{center}\begin{minipage}{75.00mm}
\setlength{\unitlength}{0.0160mm}\begin{picture}(4688,4688)

  % courbe 1
  \qbezier[0](4625,1798)(4836,2740)(4343,3571)
  \qbezier[0](4342,3570)(3962,4181)(3308,4481)
  \qbezier[0](3308,4481)(2830,4695)(2306,4688)
  \qbezier[0](2306,4689)(1465,4666)(820,4126)
  \qbezier[0](819,4127)(-28,3369)(0,2233)
  \qbezier[0](2,2232)(24,1786)(208,1378)
  \qbezier[0](208,1378)(269,1244)(346,1118)
  \qbezier[0](346,1118)(726,507)(1380,207)
  \qbezier[0](1380,207)(1858,-7)(2382,0)
  \qbezier[0](2382,-1)(3223,22)(3868,562)
  \qbezier[0](3868,562)(4441,1060)(4624,1798)

  % courbe 2
  \qbezier[0](2344,2344)(2063,3028)(1782,3713)

  % courbe 3
  \qbezier[0](2344,2344)(1611,2444)(878,2543)

  % courbe 4
  \qbezier[0](1449,3986)(1382,3998)(1310,3993)
  \qbezier[0](1310,3993)(1248,3989)(1184,3972)
  \qbezier[0](1184,3972)(1142,3960)(1100,3943)
  \qbezier[0](1098,3945)(632,3731)(494,3222)
  \qbezier[0](496,3220)(486,3175)(481,3130)
  \qbezier[0](481,3130)(475,3074)(477,3020)
  \qbezier[0](477,3020)(480,2955)(495,2896)
  \qbezier[0](495,2896)(514,2817)(554,2751)
  \qbezier[0](554,2751)(594,2687)(651,2640)
  \qbezier[0](651,2640)(715,2587)(797,2561)
  \qbezier[0](797,2561)(870,2538)(951,2538)
  \qbezier[0](951,2538)(1012,2538)(1075,2551)
  \qbezier[0](1075,2551)(1128,2562)(1181,2582)
  \qbezier[0](1183,2580)(1678,2795)(1812,3333)
  \qbezier[0](1810,3335)(1819,3381)(1822,3426)
  \qbezier[0](1822,3426)(1827,3481)(1823,3535)
  \qbezier[0](1823,3535)(1818,3598)(1801,3657)
  \qbezier[0](1801,3657)(1782,3724)(1747,3782)
  \qbezier[0](1747,3782)(1644,3947)(1449,3986)

  % courbe 5
  \qbezier[0](2344,2344)(1709,2835)(1074,3326)

  % courbe 6
  \qbezier[0](1836,2737)(1780,2588)(1758,2424)

  % courbe 7
  \qbezier[0](2344,2344)(2705,2919)(3066,3494)

  % courbe 8
  \qbezier[0](2344,2344)(1983,2919)(1622,3494)

  % courbe 9
  \qbezier[0](2968,3374)(3155,3555)(3145,3786)
  \qbezier[0](3145,3786)(3140,3874)(3104,3956)
  \qbezier[0](3104,3956)(3075,4021)(3030,4079)
  \qbezier[0](3030,4079)(2990,4128)(2939,4171)
  \qbezier[0](2939,4171)(2897,4207)(2848,4238)
  \qbezier[0](2848,4240)(2394,4490)(1907,4278)
  \qbezier[0](1907,4275)(1865,4254)(1826,4229)
  \qbezier[0](1826,4229)(1788,4204)(1753,4175)
  \qbezier[0](1753,4175)(1702,4133)(1662,4083)
  \qbezier[0](1662,4083)(1609,4018)(1579,3943)
  \qbezier[0](1579,3943)(1545,3860)(1542,3772)
  \qbezier[0](1542,3772)(1540,3695)(1564,3619)
  \qbezier[0](1564,3619)(1584,3553)(1622,3492)
  \qbezier[0](1622,3492)(1656,3440)(1701,3393)
  \qbezier[0](1701,3393)(1739,3354)(1783,3321)
  \qbezier[0](1783,3321)(1820,3294)(1860,3270)
  \qbezier[0](1860,3268)(2325,3029)(2802,3253)
  \qbezier[0](2802,3256)(2843,3278)(2881,3304)
  \qbezier[0](2881,3304)(2928,3336)(2967,3374)

  % courbe 10
  \qbezier[0](2344,2344)(2344,3097)(2344,3851)

  % courbe 11
  \qbezier[0](2344,2947)(2493,2892)(2633,2804)

  \put(1000,3400){\mbox{$\vec{u}$}}
  \put(1600,2600){\mbox{$\alpha$}}
  \put(2400,3800){\mbox{$\vec{v}$}}
  \put(2600,3000){\mbox{$\beta$}}
\end{picture}\end{minipage}\end{center}
\end{figure}
On va chercher � exprimer les coordonn�es des intersections dans le
rep�re non orthonorm� $\vec{u}$, $\vec{v}$, $\vec{u}\wedge\vec{v}$
($\vec{u}\wedge\vec{v}$ existe d�s lors que les c�nes ne sont pas
coaxiaux).
\begin{displaymath}
\vec{p} = a\vec{u} + b\vec{v} + c \vec{u} \wedge \vec{v}
\end{displaymath}
Les conditions
\begin{eqnarray*}
\vec{p}\cdot\vec{u} & = & \cos\alpha \\
\vec{p}\cdot\vec{v} & = & \cos\beta \\
\vec{p}\cdot\vec{p} & = & 1
\end{eqnarray*}
permettent d'exprimer les coordonn�es :
\begin{eqnarray*}
a & = & \frac{\cos\alpha - \vec{u}\cdot\vec{v} \cos\beta}
             {1-(\vec{u}\cdot\vec{v})^2} \\
b & = & \frac{\cos\beta - \vec{u}\cdot\vec{v} \cos\alpha}
             {1-(\vec{u}\cdot\vec{v})^2} \\
c & = & \pm\sqrt{\frac{1-(a\vec{u} + b\vec{v})^2}
                      {1-(\vec{u}\cdot\vec{v})^2}}
\end{eqnarray*}
La condition d'existence de $c$ ($(a\vec{u} + b\vec{v}) < 1$) est la
condition d'existence des intersections.

Les deux choix possibles de $c$ donnent les deux solutions quand elles
existent, sym�triques par rapport au plan $(\vec{u},~\vec{v})$. On
notera les solutions $\vec{p}_+$ et $\vec{p}_-$ selon le signe de $c$
dans la suite de cette annexe.
\begin{figure}[htbp]\caption{\label{fig:variation}variation des intersections}
\begin{center}\begin{minipage}{75.00mm}
\setlength{\unitlength}{0.0160mm}\begin{picture}(4688,4688)

  % courbe 1
  \qbezier[0](4625,1798)(4836,2740)(4343,3571)
  \qbezier[0](4342,3570)(3962,4181)(3308,4481)
  \qbezier[0](3308,4481)(2830,4695)(2306,4688)
  \qbezier[0](2306,4689)(1465,4666)(820,4126)
  \qbezier[0](819,4127)(-28,3369)(0,2233)
  \qbezier[0](2,2232)(24,1786)(208,1378)
  \qbezier[0](208,1378)(269,1244)(346,1118)
  \qbezier[0](346,1118)(726,507)(1380,207)
  \qbezier[0](1380,207)(1858,-7)(2382,0)
  \qbezier[0](2382,-1)(3223,22)(3868,562)
  \qbezier[0](3868,562)(4441,1060)(4624,1798)

  % courbe 2
  \qbezier[0](2344,2344)(1709,2835)(1074,3326)

  % courbe 3
  \qbezier[0](2344,2344)(2063,3028)(1782,3713)

  % courbe 4
  \qbezier[0](2344,2344)(1611,2444)(878,2543)

  % courbe 5
  \qbezier[0](1449,3986)(1382,3998)(1310,3993)
  \qbezier[0](1310,3993)(1248,3989)(1184,3972)
  \qbezier[0](1184,3972)(1142,3960)(1100,3943)
  \qbezier[0](1098,3945)(632,3731)(494,3222)
  \qbezier[0](496,3220)(486,3175)(481,3130)
  \qbezier[0](481,3130)(475,3074)(477,3020)
  \qbezier[0](477,3020)(480,2955)(495,2896)
  \qbezier[0](495,2896)(514,2817)(554,2751)
  \qbezier[0](554,2751)(594,2687)(651,2640)
  \qbezier[0](651,2640)(715,2587)(797,2561)
  \qbezier[0](797,2561)(870,2538)(951,2538)
  \qbezier[0](951,2538)(1012,2538)(1075,2551)
  \qbezier[0](1075,2551)(1128,2562)(1181,2582)
  \qbezier[0](1183,2580)(1678,2795)(1812,3333)
  \qbezier[0](1810,3335)(1819,3381)(1822,3426)
  \qbezier[0](1822,3426)(1827,3481)(1823,3535)
  \qbezier[0](1823,3535)(1818,3598)(1801,3657)
  \qbezier[0](1801,3657)(1782,3724)(1747,3782)
  \qbezier[0](1747,3782)(1644,3947)(1449,3986)

  % courbe 6
  \qbezier[0](2344,2344)(2173,3019)(2003,3695)

  % courbe 7
  \qbezier[0](2344,2344)(1647,2339)(951,2334)

  % courbe 8
  \qbezier[0](1562,4122)(1479,4138)(1390,4132)
  \qbezier[0](1390,4132)(1313,4126)(1235,4105)
  \qbezier[0](1235,4105)(1182,4091)(1130,4070)
  \qbezier[0](1127,4072)(552,3808)(382,3179)
  \qbezier[0](384,3177)(372,3121)(366,3065)
  \qbezier[0](366,3065)(358,2996)(361,2929)
  \qbezier[0](361,2929)(365,2849)(383,2775)
  \qbezier[0](383,2775)(407,2678)(456,2597)
  \qbezier[0](456,2597)(511,2507)(593,2445)
  \qbezier[0](593,2445)(676,2384)(778,2356)
  \qbezier[0](778,2356)(859,2334)(946,2334)
  \qbezier[0](946,2334)(1022,2334)(1099,2350)
  \qbezier[0](1099,2350)(1165,2363)(1230,2387)
  \qbezier[0](1230,2387)(1283,2407)(1334,2432)
  \qbezier[0](1337,2431)(1920,2760)(2023,3400)
  \qbezier[0](2021,3403)(2028,3472)(2025,3539)
  \qbezier[0](2025,3539)(2022,3618)(2003,3692)
  \qbezier[0](2003,3692)(1979,3789)(1930,3870)
  \qbezier[0](1930,3870)(1803,4075)(1562,4123)

  % courbe 9
  \qbezier[0](878,2543)(911,2439)(951,2334)

  % courbe 10
  \qbezier[0](2344,2344)(2410,2931)(2475,3518)

  % courbe 11
  \qbezier[0](2344,2344)(2661,2503)(2979,2662)

  % courbe 12
  \qbezier[0](2344,2344)(2070,2569)(1795,2795)

  % courbe 13
  \qbezier[0](3177,2795)(3409,3001)(3447,3291)
  \qbezier[0](3447,3291)(3463,3426)(3430,3559)
  \qbezier[0](3430,3559)(3408,3653)(3362,3740)
  \qbezier[0](3362,3740)(3323,3815)(3269,3882)
  \qbezier[0](3269,3882)(3224,3939)(3169,3988)
  \qbezier[0](3169,3988)(3126,4028)(3077,4063)
  \qbezier[0](3077,4065)(2582,4390)(2023,4190)
  \qbezier[0](2023,4188)(1967,4166)(1915,4137)
  \qbezier[0](1915,4137)(1851,4102)(1793,4058)
  \qbezier[0](1793,4058)(1724,4005)(1669,3941)
  \qbezier[0](1669,3941)(1594,3856)(1547,3757)
  \qbezier[0](1547,3757)(1494,3646)(1479,3525)
  \qbezier[0](1479,3525)(1466,3418)(1485,3310)
  \qbezier[0](1485,3310)(1501,3216)(1540,3127)
  \qbezier[0](1540,3127)(1573,3050)(1623,2980)
  \qbezier[0](1623,2980)(1664,2921)(1715,2869)
  \qbezier[0](1715,2869)(1755,2827)(1802,2790)
  \qbezier[0](1802,2790)(1848,2752)(1899,2720)
  \qbezier[0](1899,2718)(2416,2420)(2958,2650)
  \qbezier[0](2958,2652)(3012,2678)(3061,2709)
  \qbezier[0](3061,2709)(3123,2748)(3177,2795)

  % courbe 14
  \qbezier[0](1999,4023)(1717,3849)(1671,3547)

  % courbe 15
  \qbezier[0](1672,3547)(1666,3630)(1660,3714)

  % courbe 16
  \qbezier[0](1660,3714)(1715,3706)(1771,3697)

  % courbe 17
  \qbezier[0](1789,3694)(1731,3621)(1672,3547)

  % courbe 18
  \qbezier[0](1721,3218)(1816,3008)(2029,2879)

  % courbe 19
  \qbezier[0](1721,3218)(1742,3124)(1764,3029)

  % courbe 20
  \qbezier[0](1764,3029)(1812,3067)(1860,3105)

  % courbe 21
  \qbezier[0](1860,3105)(1791,3162)(1721,3218)

  \put(1000,3400){\mbox{$\vec{u}$}}
  \put(600,2400){\mbox{$\delta\alpha$}}
  \put(2500,3600){\mbox{$\vec{v}$}}
  \put(2100,4000){\mbox{$\vec{p}_+$}}
  \put(1950,2700){\mbox{$\vec{p}_-$}}

\end{picture}\end{minipage}\end{center}
\end{figure}

On cherche maintenant � savoir quelle est la partie du c�ne
($\vec{v}$,~$\beta$) qui est incluse dans le c�ne
($\vec{u}$,~$\alpha$) et quelle est la partie qui est hors du c�ne. On
fait donc varier l'angle d'ouverture $\alpha$ et on observe le sens de
variation de $\vec{p}_+(\alpha)$ et $\vec{p}_-(\alpha)$ (voir
figure~\ref{fig:variation}).

Les d�riv�es des coordonn�es sont :
\begin{eqnarray*}
\frac{\partial a}{\partial\alpha}
  & = & \frac{-\sin\alpha}
             {1-(\vec{u}\cdot\vec{v})^2} \\
\frac{\partial b}{\partial\alpha}
  & = & \frac{(\vec{u}\cdot\vec{v})\sin\alpha}
             {1-(\vec{u}\cdot\vec{v})^2} \\
\frac{\partial[\|a\vec{u}+b\vec{v}\|^2]}{\partial\alpha}
  & = & \frac{2\sin\alpha}{1-(\vec{u}\cdot\vec{v})^2}
        [a\vec{u}+b\vec{v}]\cdot[-\vec{u}+(\vec{u}\cdot\vec{v})\vec{v}]
    = -2a\sin\alpha \\
\frac{\partial c_+}{\partial\alpha}
  & = & \frac{a\sin\alpha}{c_+\left(1-(\vec{u}\cdot\vec{v})^2\right)} \\
\frac{\partial c_-}{\partial\alpha}
  & = & \frac{a\sin\alpha}{c_-\left(1-(\vec{u}\cdot\vec{v})^2\right)} \\
\end{eqnarray*}
On en d�duit :
\begin{eqnarray*}
\frac{\partial\vec{p}_+}{\partial\alpha}
  & = & \frac{\sin\alpha}{1-(\vec{u}\cdot\vec{v})^2}
        \left[-\vec{u}
              +(\vec{u}\cdot\vec{v})\vec{v}
              +\frac{a}{c_+}\vec{u}\wedge\vec{v}
        \right] \\
\frac{\partial\vec{p}_-}{\partial\alpha}
  & = & \frac{\sin\alpha}{1-(\vec{u}\cdot\vec{v})^2}
        \left[-\vec{u}
              +(\vec{u}\cdot\vec{v})\vec{v}
              +\frac{a}{c_-}\vec{u}\wedge\vec{v}
        \right] \\
\end{eqnarray*}
� ce stade, on peut remarquer que :
\begin{eqnarray*}
\vec{v}\wedge\vec{p}_+
 & = & a\vec{v}\wedge\vec{u}
      +b\vec{v}\wedge\vec{v}
      +c_+\vec{v}\wedge(\vec{u}\wedge\vec{v})\\
 & = & c_+\left[\vec{u}-(\vec{u}\cdot\vec{v})\vec{v}\right]
      -a\vec{u}\wedge\vec{v}
\end{eqnarray*}
(par application du double produit vectoriel : $\vec{a} \wedge
(\vec{b} \wedge \vec{c}) = (\vec{a} \cdot \vec{c}) \vec{b} - (\vec{a}
\cdot \vec{b}) \vec{c})$

De m�me, on d�montre :
\begin{displaymath}
\vec{v}\wedge\vec{p}_-
 = c_-\left[\vec{u}-(\vec{u}\cdot\vec{v})\vec{v}\right]
   -a\vec{u}\wedge\vec{v}
\end{displaymath}
Donc :
\begin{eqnarray*}
\frac{\partial\vec{p}_+}{\partial\alpha}
  & = & \frac{-\sin\alpha}{c_+}
        \frac{\vec{v}\wedge\vec{p}_+}
             {1-(\vec{u}\cdot\vec{v})^2} \\
\frac{\partial\vec{p}_-}{\partial\alpha}
  & = & \frac{-\sin\alpha}{c_-}
        \frac{\vec{v}\wedge\vec{p}_-}
             {1-(\vec{u}\cdot\vec{v})^2} \\
\end{eqnarray*}
Comme $\sin(\alpha) > 0$, $c_+ > 0$, $c_- < 0$,
$1-(\vec{u}\cdot\vec{v})^2 > 0$, une diminution de l'angle d'ouverture
$\alpha$ engendre une rotation n�gative de $\vec{p}_-$ autour de
$\vec{v}$, et une rotation positive de $\vec{p}_+$ autour de
$\vec{v}$.

La partie du c�ne ($\vec{v}$,~$\beta$) incluse dans le c�ne
($\vec{u}$,~$\alpha$) est donc la partie qui va de $\vec{p}_+$ vers
$\vec{p}_-$ en tournant positivement autour de $\vec{v}$ (voir
figure~\ref{fig:inclusion}). Un raisonnement sym�trique montre que la
partie du c�ne ($\vec{u}$,~$\alpha$) incluse dans le c�ne
($\vec{v}$,~$\beta$) est la partie qui va de $\vec{p}_-$ vers
$\vec{p}_+$ en tournant positivement autour de $\vec{u}$.
\begin{figure}[htbp]\caption{\label{fig:inclusion}parties incluses}
\begin{center}\begin{minipage}{75.00mm}
\setlength{\unitlength}{0.0160mm}\begin{picture}(4688,4688)

  % courbe 1
  \qbezier[0](4625,1798)(4836,2740)(4343,3571)
  \qbezier[0](4342,3570)(3962,4181)(3308,4481)
  \qbezier[0](3308,4481)(2830,4695)(2306,4688)
  \qbezier[0](2306,4689)(1465,4666)(820,4126)
  \qbezier[0](819,4127)(-28,3369)(0,2233)
  \qbezier[0](2,2232)(24,1786)(208,1378)
  \qbezier[0](208,1378)(269,1244)(346,1118)
  \qbezier[0](346,1118)(726,507)(1380,207)
  \qbezier[0](1380,207)(1858,-7)(2382,0)
  \qbezier[0](2382,-1)(3223,22)(3868,562)
  \qbezier[0](3868,562)(4441,1060)(4624,1798)

  % courbe 2
  \qbezier[0](2344,2344)(1869,3084)(1393,3823)

  % courbe 3
  \qbezier[0](2344,2344)(2325,3159)(2306,3974)

  % courbe 4
  \qbezier[0](2344,2344)(1610,2700)(877,3056)

  % courbe 5
  \qbezier[0](2214,3631)(2385,3962)(2239,4203)
  \qbezier[0](2238,4203)(2186,4283)(2099,4335)
  \qbezier[0](2099,4335)(2037,4372)(1963,4392)
  \qbezier[0](1963,4392)(1899,4409)(1828,4414)
  \qbezier[0](1828,4414)(1768,4418)(1705,4414)
  \qbezier[0](1703,4416)(1135,4347)(777,3871)
  \qbezier[0](778,3868)(748,3825)(724,3780)
  \qbezier[0](724,3780)(694,3725)(673,3669)
  \qbezier[0](673,3669)(649,3602)(638,3536)
  \qbezier[0](638,3536)(625,3450)(638,3370)
  \qbezier[0](638,3370)(652,3281)(699,3209)
  \qbezier[0](699,3209)(736,3153)(790,3110)
  \qbezier[0](790,3110)(837,3073)(894,3048)
  \qbezier[0](894,3048)(952,3022)(1018,3009)
  \qbezier[0](1018,3009)(1074,2998)(1134,2996)
  \qbezier[0](1134,2996)(1183,2994)(1233,2998)
  \qbezier[0](1234,2996)(1803,3065)(2161,3541)
  \qbezier[0](2160,3543)(2189,3587)(2214,3631)

  % courbe 6
  \qbezier[0](1894,3124)(2411,3502)(2430,3964)
  \qbezier[0](2429,3965)(2428,4013)(2420,4058)
  \qbezier[0](2420,4058)(2412,4103)(2397,4144)
  \qbezier[0](2397,4144)(2378,4194)(2348,4238)

  % courbe 7
  \qbezier[0](2348,4238)(2417,4138)(2485,4039)

  % courbe 8
  \qbezier[0](2485,4039)(2424,4037)(2363,4034)

  % courbe 9
  \qbezier[0](2363,4034)(2355,4136)(2348,4238)

  % courbe 10
  \qbezier[0](2344,2344)(2403,3017)(2461,3690)

  % courbe 11
  \qbezier[0](2344,2344)(2743,2708)(3142,3071)

  % courbe 12
  \qbezier[0](2344,2344)(2014,2771)(1684,3198)

  % courbe 13
  \qbezier[0](3136,3066)(3331,3243)(3363,3484)
  \qbezier[0](3362,3484)(3376,3602)(3345,3719)
  \qbezier[0](3345,3719)(3320,3812)(3268,3898)
  \qbezier[0](3268,3898)(3229,3962)(3177,4020)
  \qbezier[0](3177,4020)(3132,4068)(3080,4110)
  \qbezier[0](3080,4113)(2592,4464)(2015,4257)
  \qbezier[0](2014,4254)(1963,4233)(1917,4206)
  \qbezier[0](1917,4206)(1858,4174)(1807,4134)
  \qbezier[0](1807,4134)(1746,4086)(1697,4028)
  \qbezier[0](1697,4028)(1640,3962)(1602,3886)
  \qbezier[0](1602,3886)(1554,3788)(1541,3683)
  \qbezier[0](1541,3683)(1530,3589)(1548,3495)
  \qbezier[0](1548,3495)(1563,3412)(1600,3334)
  \qbezier[0](1600,3334)(1632,3267)(1678,3206)
  \qbezier[0](1678,3206)(1724,3145)(1783,3091)
  \qbezier[0](1783,3091)(1822,3056)(1866,3024)
  \qbezier[0](1867,3021)(2380,2695)(2939,2932)
  \qbezier[0](2940,2935)(2988,2959)(3033,2988)
  \qbezier[0](3033,2988)(3088,3023)(3136,3066)

  % courbe 14
  \qbezier[0](1921,4292)(1467,4066)(1423,3629)
  \qbezier[0](1424,3628)(1410,3384)(1555,3167)

  % courbe 15
  \qbezier[0](1555,3167)(1485,3237)(1415,3307)

  % courbe 16
  \qbezier[0](1415,3307)(1469,3334)(1523,3361)

  % courbe 17
  \qbezier[0](1523,3361)(1539,3264)(1555,3167)

  \put(1200,3900){\mbox{$\vec{u}$}}
  \put(2600,3800){\mbox{$\vec{v}$}}
  \put(2200,4400){\mbox{$\vec{p}_+$}}
  \put(1600,2900){\mbox{$\vec{p}_-$}}
\end{picture}\end{minipage}\end{center}
\end{figure}
Ce r�sultat est vrai dans toutes les configurations de $\vec{u}$,
$\vec{v}$, $\alpha$, $\beta$.
