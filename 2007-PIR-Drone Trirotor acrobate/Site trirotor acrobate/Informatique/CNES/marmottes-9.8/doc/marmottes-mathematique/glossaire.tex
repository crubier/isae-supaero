% -*- mode: latex; tex-main-file: "marmottes-mathematique.tex" -*-
% $Id: glossaire.tex,v 1.4 2001/01/26 17:14:44 luc Exp $

\Glossaire{angles de \textsc{Cardan}}{Les angles de \textsc{Cardan}
permettent de d�finir une rotation quelconque en dimension 3 par la
composition de trois rotations successives autour des trois axes
canoniques dans un ordre qui varie selon les conventions :
$\vec{\imath}$ puis $\vec{\jmath}$ puis $\vec{k}$, ou $\vec{k}$ puis
$\vec{\imath}$ puis $\vec{\jmath}$, ou n'importe quel autre ordre.}

\Glossaire{angles d'\textsc{Euler}}{Les angles d'\textsc{Euler}
permettent de d�finir une rotation quelconque en dimension 3 par la
composition de trois rotations successives autour de deux axes
canoniques, la premi�re et la troisi�me rotation se faisant autour
d'un axe et la seconde rotation se faisant autour de l'autre axe. La
convention la plus courante consiste � utiliser les axes $\vec{k}$ et
$\vec{\imath}$ dans l'ordre : $\vec{k}$ puis $\vec{\imath}$ puis
$\vec{k}$.}

\Glossaire{\textsc{aocs}}{Attitude and Orbit Control System}

\Glossaire{barbecue}{Mode de pilotage consistant � pointer un axe
satellite sur le Soleil et � tourner � vitesse constante autour de cet
axe. Ce mode est un mode d'attente s�r pour le satellite, il est
souvent utilis� comme mode de survie ou comme mode d'attente hors
mission, par exemple pendant les phases de mise � poste.}

\Glossaire{\textsc{cantor}}{Composants et Algorithmes Num�riques Traduits sous
forme d'Objets R�utilisables, biblioth�que math�matique utilis�e par
\bibliotheque{marmottes}}

\Glossaire{capteur}{�quipement bord permettant � un satellite de
mesurer son mouvement autour de son centre de gravit�}

\Glossaire{mouvement de \emph{coning}}{Mode de pilotage consistant � faire
parcourir � un axe satellite un c�ne autour du Soleil � une certaine
vitesse angulaire et � tourner � la m�me vitesse autour de cet axe. Ce
mode est tr�s souvent utilis� pour la recherche de la Terre lors du
passage d'un pointage Soleil � un pointage Terre.}

\Glossaire{\textsc{marmottes}}{Mod�lisation d'Attitude par R�cup�ration des
Mesures d'Orientation pour Tout Type d'Engin Spatial, biblioth�que de
simulation de l'orientation des satellite}

\Glossaire{\textsc{scao}}{Syst�me de Contr�le d'Attitude et d'Orbite}
