% -*- mode: latex; tex-main-file: "marmottes-mathematique.tex" -*-
% $Id: conclusion.tex,v 1.1 2001/01/26 16:53:58 luc Exp $

\section{Conclusion}\label{sec:conclusion}
Les m�thodes qui ont �t� d�crites permettent de r�aliser des
simulateurs d'�volution d'orientation dans la composante sol des
syst�mes spatiaux. Elles ne r�pondent pas � tous les besoins li�s �
l'orientation des satellites. Ni les erreurs de mesures ni les
actuateurs ne sont pris en compte ; les �tudes n�cessitant leur
mod�lisation doivent donc ajouter leurs propres algorithmes, m�me si
elles peuvent utiliser �galement la base pr�sent�e. Ces m�thodes ne
sont pas non plus utilisables � bord, elles sont essentiellement
orient�es vers de la simulation.

Ces m�thodes s'appuient sur une description de
l'orientation des satellites tr�s simple et proche des concepts
physiques familiers aux ing�nieurs en m�canique spatiale. Cette
description permet de s�parer clairement la description et la
d�termination de l'orientation, ce qui permet la mise au point
d'outils multi-missions.

Ces m�thodes sont impl�ment�es dans une biblioth�que valid�e
op�rationnellement. De nombreux projets en ont b�n�fici� depluis
plusieurs ann�es. Cette biblioth�que est diffus�e librement par le
\textsc{cnes}\footnote{\texttt{http\char58//logiciels.cnes.fr}}.
