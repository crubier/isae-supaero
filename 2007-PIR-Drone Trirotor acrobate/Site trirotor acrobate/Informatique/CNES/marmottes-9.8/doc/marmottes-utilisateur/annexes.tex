% -*- mode: latex; tex-main-file: "marmottes-utilisateur.tex" -*-
% $Id: annexes.tex,v 1.5 2003/02/13 09:49:37 marmottes Exp $
\clearpage\section{R�ordonnancement des senseurs}\label{sec:reordonnancement}
Pour r�soudre l'attitude, Marmottes regroupe deux senseurs de m�me
genre (cin�matiques ou g�om�triques) et utilise le troisi�me senseur
pour annuler une fonction (avec un seuil de convergence d�pendant de
la pr�cision de ce troisi�me senseur).

Afin de permettre � l'utilisateur de savoir quel senseur est isol�
parmi les trois senseurs de consigne, voici l'algorithme utilis� par
Marmottes.

Soient s1, s2 et s3 les senseurs dans l'ordre utilisateur

Soient sa1, sa2 et sb les senseurs dans l'ordre de r�solution

\begin{tabbing}
Si \=(s1 et s2 sont de m�me type)\+\\
sa1 = s1\\
sa2 = s2\\
sb = s3\-\\
Sinon\+\\
Si \=(s1 et s3 sont de m�me type)\+\\
sa1 = s1\\
sa2 = s3\\
sb = s2\-\\
Sinon\+\\
sa1 = s2\\
sa2 = s3\\
sb = s1\-\\
finsi\-\\
finsi
\end{tabbing}

\clearpage\section{exemple de fichier senseurs en francais}
\newlength{\largeurfrancais}
\settowidth{\largeurfrancais}{\ttfamily
xxxxxxxxxxxxxxxxxxxxxxxxxxxxxxxxxxxxxxxxxxxxxxxxxxxxxxxxxxxxxxxxxxxxxxxxxxxx
}\begin{center}\begin{minipage}{\largeurfrancais}\begin{verbatim}
# Senseur solaire 1 (tangage)
SOLAIRE_1_TANGAGE
{ type                {diedre}
  cible               {soleil}
  precision           { 0.01 }

  # d�finition des axes senseurs en rep�re satellite
  repere { i { -1  0  0 } j { 0 0 1 } k { 0 1 0 }}

  # d�finition des axes particuliers en rep�re senseur
  axe_calage      { 0 1 0 }
  axe_sensible    { 0 0 1 }
  reference_zero  { 1 0 0 }


  # d�finition du champ de vue (vecteurs not�s angulairement)
  champ_de_vue
  { { # di�dre d'axe j senseur (ouverture +/- 32 degr�s)
      { cone { axe { 0.0  58.0 } angle { 90 }}}
      inter
      { cone { axe { 0.0 -58.0 } angle { 90 }}}
    }
    inter
    { # di�dre d'axe k senseur (ouverture +/- 32 degr�s)
      { cone { axe {  58.0 0.0 } angle { 90 }}}
      inter
      { cone { axe { -58.0 0.0 } angle { 90 }}}
    }
  }
}

# Senseur solaire 1 (lacet)
SOLAIRE_1_YAW
{ => { SOLAIRE_1_TANGAGE}

  # seuls les axes de mesure diff�rent entre SOLAIRE_1_TANGAGE SOLAIRE_1_YAW
  axe_calage      { 0 0 1 }
  axe_sensible    { 0 1 0 }
  reference_zero  { 1 0 0 }

}
\end{verbatim}\end{minipage}\end{center}
\begin{center}\begin{minipage}{\largeurfrancais}\begin{verbatim}
IRES_ROLL
{ type           { limbe }
  precision      { 0.2 }

  # le rep�re IRES s'obtient en tournant le rep�re satellite
  # de -0.4702 degr�s autour de l'axe Ysat
  repere         { axe { 0 1 0 } angle { -0.4702} }

  axe_sensible   { -1 0 0 }
  reference_zero {  0 0 1 }

  # l'�l�ment de d�tection de l'IRES est un bolom�tre carr� de 1.3 degr�s
  # de largeur tourn� de 45 degr�s autour de Z : c'est un double di�dre
  bolometre_fictif { { { cone { axe {  45 0.65 } angle { 90 } } }
                       inter
                       { cone { axe { 225 0.65 } angle { 90 } } }
                     }
                     inter
                     { { cone { axe { 135 0.65 } angle { 90 } } }
                       inter
                       { cone { axe { 315 0.65 } angle { 90 } } }
                     }
                   }

  # pour d�finir un scan, on place un bolom�tre au milieu du scan
  # on le d�place de -1/2 scan, puis on le tra�ne le long du scan
  # l'angle total vaut 8 degr�s en champ large, 5.3 degr�s en champ �troit
  centre_scan_1 { rotation { axe { 0 1 0 } angle { -6.2 } }
                  de       { axe { 1 0 0 } angle {  6.2 } }
                }
  centre_scan_2 { rotation { axe { 0 1 0 } angle { -6.2 } }
                  de       { axe { 1 0 0 } angle { -6.2 } }
                }
  centre_scan_3 { rotation { axe { 0 1 0 } angle {  6.2 } }
                  de       { axe { 1 0 0 } angle {  6.2 } }
                }
  centre_scan_4 { rotation { axe { 0 1 0 } angle {  6.2 } }
                  de       { axe { 1 0 0 } angle { -6.2 } }
                }
  scan          { axe { 0 1 0 } angle {  8.00 } }
  demi_scan     { axe { 0 1 0 } angle { -4.00 } }
\end{verbatim}\end{minipage}\end{center}
\begin{center}\begin{minipage}{\largeurfrancais}\begin{verbatim}
  # scans �l�mentaires du champ de vue
  scan_1
  { balayage { => { IRES_ROLL.scan } }
    de       { rotation { => { IRES_ROLL.demi_scan } }
               de       { rotation { => { IRES_ROLL.centre_scan_1 } }
                          de       { => { IRES_ROLL.bolometre_fictif } }
                        }
             }
  }

  scan_2
  { balayage { => { IRES_ROLL.scan } }
    de       { rotation { => { IRES_ROLL.demi_scan } }
               de       { rotation { => { IRES_ROLL.centre_scan_2 } }
                          de       { => { IRES_ROLL.bolometre_fictif } }
                        }
             }
  }

  scan_3
  { balayage { => { IRES_ROLL.scan } }
    de       { rotation { => { IRES_ROLL.demi_scan } }
               de       { rotation { => { IRES_ROLL.centre_scan_3 } }
                          de       { => { IRES_ROLL.bolometre_fictif } }
                        }
             }
  }

  scan_4
  { balayage { => { IRES_ROLL.scan } }
    de       { rotation { => { IRES_ROLL.demi_scan } }
               de       { rotation { => { IRES_ROLL.centre_scan_4 } }
                          de       { => { IRES_ROLL.bolometre_fictif } }
                        }
             }
  }
\end{verbatim}\end{minipage}\end{center}
\begin{center}\begin{minipage}{\largeurfrancais}\begin{verbatim}
  champ_de_vue
  { { { => { IRES_ROLL.scan_1 } } et { => { IRES_ROLL.scan_2 } } }
    ou
    { { => { IRES_ROLL.scan_3 } } et { => { IRES_ROLL.scan_4 } } }
  }

  champ_d_inhibition_soleil
  { { marge { 3.0 }
      sur   { { { => { IRES_ROLL.scan_1 } }
                union
                { => { IRES_ROLL.scan_2 } }
              }
              union
              { { => { IRES_ROLL.scan_3 } }
                union
                { => { IRES_ROLL.scan_4 } }
              }
            }
    }
    sauf
    { cone { axe { 0 0 1 } angle { 8.2 } } }
  }

  champ_d_inhibition_lune { => { IRES_ROLL.champ_d_inhibition_soleil } }

}

IRES_PITCH
{ axe_sensible { 0 -1 0 }

  champ_de_vue
  { { { => { IRES_ROLL.scan_1 } } et { => { IRES_ROLL.scan_3 } } }
    ou
    { { => { IRES_ROLL.scan_2 } } et { => { IRES_ROLL.scan_4 } } }
  }

  => {IRES_ROLL}

}
\end{verbatim}\end{minipage}\end{center}
\begin{center}\begin{minipage}{\largeurfrancais}\begin{verbatim}
AEF_Ascension   { type      { ascension_droite }
                  precision { 0.001 }
                  repere    { i { 1 0 0 } j { 0 1 0 } k { 0 0 1 } }
                  observe   { 0 0 1 }
                }
 
AEF_Declination { => {AEF_Ascension} type { declinaison }}

COMMUNS-PSEUDOS
{ precision { 0.0001 }
  repere    { 1 0 0 0 } # quaternion identit�
}

ALPHA_X { type { ascension_droite } observe { 1 0 0 } => {COMMUNS-PSEUDOS}}
DELTA_X { type { declinaison }      observe { 1 0 0 } => {COMMUNS-PSEUDOS}}
ALPHA_Y { type { ascension_droite } observe { 0 1 0 } => {COMMUNS-PSEUDOS}}
DELTA_Y { type { declinaison }      observe { 0 1 0 } => {COMMUNS-PSEUDOS}}
ALPHA_Z { type { ascension_droite } observe { 0 0 1 } => {COMMUNS-PSEUDOS}}
DELTA_Z { type { declinaison }      observe { 0 0 1 } => {COMMUNS-PSEUDOS}}

GYRO_X  { type { cinematique } axe_sensible { 1 0 0 } => {COMMUNS-PSEUDOS}}
GYRO_Y  { type { cinematique } axe_sensible { 0 1 0 } => {COMMUNS-PSEUDOS}}
GYRO_Z  { type { cinematique } axe_sensible { 0 0 1 } => {COMMUNS-PSEUDOS}}
\end{verbatim}\end{minipage}\end{center}

\clearpage\section{exemple de fichier senseurs en anglais}
\newlength{\largeuranglais}
\settowidth{\largeuranglais}{\ttfamily
xxxxxxxxxxxxxxxxxxxxxxxxxxxxxxxxxxxxxxxxxxxxxxxxxxxxxxxxxxxxxxxxxxxxxxxxxxxxxx
}\begin{center}\begin{minipage}{\largeuranglais}\begin{verbatim}
# Pitch sun sensor 1
SUN_1_PITCH
{ type                {dihedral}
  target              {sun}
  accuracy            { 0.01 }

  # definition of sensor axis in satellite frame
  frame { i { -1  0  0 } j { 0 0 1 } k { 0 1 0 }}

  # definition of special vectors in sensor frame
  wedging_axis      { 0 1 0 }
  sensitive_axis    { 0 0 1 }
  zero_reference    { 1 0 0 }


  # field of view definition (vectors are described angularly)
  field of view
  { { # j sensor axis dihedra (opening +/- 32 degrees)
      { cone { axis { 0.0  58.0 } angle { 90 }}}
      inter
      { cone { axis { 0.0 -58.0 } angle { 90 }}}
    }
    inter
    { # k sensor axis dihedra (opening +/- 32 degrees)
      { cone { axis {  58.0 0.0 } angle { 90 }}}
      inter
      { cone { axis { -58.0 0.0 } angle { 90 }}}
    }
  }
}

# Yaw sun sensor 1
SUN_1_YAW
{ => { SUN_1_PITCH}

  # only measurements axis differ from SUN_1_PITCH
  wedging_axis      { 0 0 1 }
  sensitive_axis    { 0 1 0 }
  zero_reference    { 1 0 0 }

}
\end{verbatim}\end{minipage}\end{center}
\begin{center}\begin{minipage}{\largeuranglais}\begin{verbatim}
IRES_ROLL
{ type           { limb }
  accuracy       { 0.2 }

  # IRES frame is satellite frame rotated
  # -0.4702 degrees around Ysat
  frame            { axis { 0 1 0 } angle { -0.4702} }

  sensitive_axis   { -1 0 0 }
  zero_reference   {  0 0 1 }

  # IRES detector is a 1.3 degrees square bolometer
  # rotated by 45 degrees around Z : it is a double-dihedra
  fictious_bolometer { { { cone { axis {  45 0.65 } angle { 90 } } }
                         inter
                         { cone { axis { 225 0.65 } angle { 90 } } }
                       }
                       inter
                       { { cone { axis { 135 0.65 } angle { 90 } } }
                         inter
                         { cone { axis { 315 0.65 } angle { 90 } } }
                       }
                     }

  # in order to define a scan, one places the bolometer at the middle
  # of the scan, then shift it -1/2 scan, and then one spread it over
  # all scan long. total angle is 8 degrees in wide scan mode and 5.3
  # degrees in narrow scan mode
  center_scan_1 { rotation { axis { 0 1 0 } angle { -6.2 } }
                  of       { axis { 1 0 0 } angle {  6.2 } }
                }
  center_scan_2 { rotation { axis { 0 1 0 } angle { -6.2 } }
                  of       { axis { 1 0 0 } angle { -6.2 } }
                }
  center_scan_3 { rotation { axis { 0 1 0 } angle {  6.2 } }
                  of       { axis { 1 0 0 } angle {  6.2 } }
                }
  center_scan_4 { rotation { axis { 0 1 0 } angle {  6.2 } }
                  of       { axis { 1 0 0 } angle { -6.2 } }
                }
  scan          { axis { 0 1 0 } angle {  8.00 } }
  half_scan     { axis { 0 1 0 } angle { -4.00 } }
\end{verbatim}\end{minipage}\end{center}
\begin{center}\begin{minipage}{\largeuranglais}\begin{verbatim}
  # elementary scans
  scan_1
  { spread { => { IRES_ROLL.scan } }
    of       { rotation { => { IRES_ROLL.half_scan } }
               of       { rotation { => { IRES_ROLL.center_scan_1 } }
                          of       { => { IRES_ROLL.fictious_bolometer } }
                        }
             }
  }

  scan_2
  { spread { => { IRES_ROLL.scan } }
    of       { rotation { => { IRES_ROLL.half_scan } }
               of       { rotation { => { IRES_ROLL.center_scan_2 } }
                          of       { => { IRES_ROLL.fictious_bolometer } }
                        }
             }
  }

  scan_3
  { spread { => { IRES_ROLL.scan } }
    of       { rotation { => { IRES_ROLL.half_scan } }
               of       { rotation { => { IRES_ROLL.center_scan_3 } }
                          of       { => { IRES_ROLL.fictious_bolometer } }
                        }
             }
  }

  scan_4
  { spread { => { IRES_ROLL.scan } }
    of       { rotation { => { IRES_ROLL.half_scan } }
               of       { rotation { => { IRES_ROLL.center_scan_4 } }
                          of       { => { IRES_ROLL.fictious_bolometer } }
                        }
             }
  }
\end{verbatim}\end{minipage}\end{center}
\begin{center}\begin{minipage}{\largeuranglais}\begin{verbatim}
  field_of_view
  { { { => { IRES_ROLL.scan_1 } } and { => { IRES_ROLL.scan_2 } } }
    or
    { { => { IRES_ROLL.scan_3 } } and { => { IRES_ROLL.scan_4 } } }
  }

  sun_field_of_inhibition
  { { margin { 3.0 }
      upon   { { { => { IRES_ROLL.scan_1 } }
                 union
                 { => { IRES_ROLL.scan_2 } }
               }
               union
               { { => { IRES_ROLL.scan_3 } }
                 union
                 { => { IRES_ROLL.scan_4 } }
               }
             }
     }
     except
     { cone { axis { 0 0 1 } angle { 8.2 } } }
  }

  moon_field_of_inhibition { => { IRES_ROLL.sun_field_of_inhibition } }

}

IRES_PITCH
{ sensitive_axis { 0 -1 0 }

  field_of_view
  { { { => { IRES_ROLL.scan_1 } } and { => { IRES_ROLL.scan_3 } } }
    or
    { { => { IRES_ROLL.scan_2 } } and { => { IRES_ROLL.scan_4 } } }
  }

  => {IRES_ROLL}

}
\end{verbatim}\end{minipage}\end{center}
\begin{center}\begin{minipage}{\largeuranglais}\begin{verbatim}
AEF_Ascension   { type      { right_ascension }
                  accuracy  { 0.001 }
                  frame     { i { 1 0 0 } j { 0 1 0 } k { 0 0 1 } }
                  observed  { 0 0 1 }
                }
 
AEF_Declination { => {AEF_Ascension} type { declination }}

PSEUDOS-COMMONS
{ accuracy { 0.0001 }
  frame    { 1 0 0 0 } # identity quaternion
}

ALPHA_X { type { right_ascension } observed { 1 0 0 } => {PSEUDOS-COMMONS}}
DELTA_X { type { declination }     observed { 1 0 0 } => {PSEUDOS-COMMONS}}
ALPHA_Y { type { right_ascension } observed { 0 1 0 } => {PSEUDOS-COMMONS}}
DELTA_Y { type { declination }     observed { 0 1 0 } => {PSEUDOS-COMMONS}}
ALPHA_Z { type { right_ascension } observed { 0 0 1 } => {PSEUDOS-COMMONS}}
DELTA_Z { type { declination }     observed { 0 0 1 } => {PSEUDOS-COMMONS}}

GYRO_X  { type { kinematic } sensitive_axis { 1 0 0 } => {PSEUDOS-COMMONS}}
GYRO_Y  { type { kinematic } sensitive_axis { 0 1 0 } => {PSEUDOS-COMMONS}}
GYRO_Z  { type { kinematic } sensitive_axis { 0 0 1 } => {PSEUDOS-COMMONS}}
\end{verbatim}\end{minipage}\end{center}

\clearpage\section{Lexique Fran�ais-Anglais des mots cl�s du fichier Senseurs}\label{sec:lexique-fr}

\begin{center}\begin{longtable}{|c|c|}
\caption{Mots-cl�s du fichier senseurs}\\
\hline
Mots cl�s en Fran\c{c}ais & Mots cl�s en Anglais\\
\hline\endfirsthead
\caption{Mots-cl�s du fichier senseurs (suite)}\\
\hline
Mots cl�s en Fran\c{c}ais & Mots cl�s en Anglais\\
\hline
\endhead
\hline\multicolumn{2}{|r|}{� suivre ...}\\\hline\endfoot
\hline\endlastfoot

altitude &
  altitude \\

angle &
   angle \\

angle\_3dB &
  angle\_3dB \\

angle\_3dB\_x &
  angle\_3dB\_x \\

angle\_3dB\_y &
  angle\_3dB\_y \\

angle\_zero &
  zero\_angle \\

axe &
   axis \\

axe\_calage &
   wedging\_axis \\

axe\_sensible &
   sensitive\_axis \\

balayage &
   spread \\

champ\_de\_vue &
   field\_of\_view \\

champ\_d\_inhibition\_corps\_central 
&
   central\_body\_field\_of\_inhibition \\

champ\_d\_inhibition\_lune &
   moon\_field\_of\_inhibition \\

champ\_d\_inhibition\_soleil &
   sun\_field\_of\_inhibition \\

cible &
   target \\

cone &
   cone \\

de &
   of \\

echantillon &
  sample \\

genre &
   kind \\

hygrometrie &
  hygrometry \\

i &
   i \\

inter &
   inter \\

j &
   j \\

k &
   k \\

longitude &
  longitude \\

latitude &
  latitude \\

marge\_eclipse\_lune
\footnote{Pour plus de pr�cisions: se reporter au \S~\ref{sec:schemaInhibition} } 
&
  moon\_eclipse\_margin \\


marge\_eclipse\_soleil
\footnote{Pour plus de pr�cisions: se reporter au \S~\ref{sec:schemaInhibition} } 
&
  sun\_eclipse\_margin \\

masque &
  mask \\

maximum &
  maximum \\

normale\_reference &
   normal\_reference \\

observe &
   observed \\

origine &
  origin \\

precision &
   accuracy \\

pression &
  pressure \\

reference &
   reference \\

reference\_zero &
   zero\_reference \\

repere &
   frame \\

rotation &
   rotation \\

sauf &
   except \\

seuil\_phase\_lune 
\footnote{Pour plus de pr�cisions: se reporter au \S~\ref{sec:schemaInhibition} } 
&
   moon\_phase\_threshold \\

temperature &
  temperature \\

type &
   type \\

union &
   union \\

v\_base &
   v\_base \\

v\_base\_1 &
   v\_base\_1 \\

v\_base\_2 &
   v\_base\_2 \\

v\_image &
   v\_image \\

v\_image\_1 &
   v\_image\_1 \\

v\_image\_2 &
   v\_image\_2 \\

\hline
\end{longtable}\end{center}

\begin{center}\begin{longtable}{|c|c|}
\caption{Types de senseurs reconnus}\\
\hline
Mots cl�s en Fran\c{c}ais & Mots cl�s en Anglais\\
\hline
\endfirsthead
\caption{Types de senseurs reconnus (suite)}\\
\hline
Mots cl�s en Fran\c{c}ais & Mots cl�s en Anglais\\
\hline
\endhead
\hline\multicolumn{2}{|r|}{� suivre ...}\\\hline\endfoot
\hline\endlastfoot

ascension\_droite &
   right\_ascension \\

cardan &
   cardan \\

cartesien &
   cartesian \\

cinematique &
   kinematic \\

declinaison &
   declination \\

diedre &
   dihedral \\

gain\_echantillonne\_1D &
   sampled\_1D\_gain \\

gain\_gauss &
   gauss\_gain \\

gain\_sinus\_cardinal\_2 &
   square\_cardinal\_sine\_gain \\

gain\_sinus\_cardinal\_xy &
   xy\_cardinal\_sine\_gain \\

gyro\_integrateur &
   integrating\_gyro \\

limbe &
   limb \\

plan\_vecteur &
   plane\_vector \\

terre &
   earth \\

vecteur &
   vector \\

\hline
\end{longtable}\end{center}

\begin{center}\begin{longtable}{|c|c|}
\caption{Types de senseurs cardans}\\
\hline
Mots cl�s en Fran\c{c}ais & Mots cl�s en Anglais\\
\hline\endfirsthead
\caption{Types de senseurs cardans (suite)}\\
\hline
Mots cl�s en Fran\c{c}ais & Mots cl�s en Anglais\\
\hline\endhead
\hline\multicolumn{2}{|r|}{� suivre ...}\\\hline\endfoot
\hline\endlastfoot

LRT-lacet &
  YRP-yaw \\

LRT-roulis &
  YRP-roll \\

LRT-tangage &
  YRP-pitch \\

LTR-lacet &
  YPR-yaw \\

LTR-roulis &
  YPR-roll \\

LTR-tangage &
  YPR-pitch \\

RLT-lacet &
  RYP-yaw \\

RLT-roulis &
  RYP-roll \\

RLT-tangage &
  RYP-pitch \\

RTL-lacet &
  RPY-yaw \\

RTL-roulis &
  RPY-roll \\

RTL-tangage &
  RPY-pitch \\

TLR-lacet &
  PYR-yaw \\

TLR-roulis &
  PYR-roll \\

TLR-tangage &
  PYR-pitch \\

TRL-lacet &
  PRY-yaw \\

TRL-roulis &
  PRY-roll \\

TRL-tangage &
  PRY-pitch \\

\hline
\end{longtable}\end{center}

\begin{center}\begin{longtable}{|c|c|}
\caption{Rep�res de r�f�rence}\\
\hline
Mots cl�s en Fran\c{c}ais & Mots cl�s en Anglais\\
\hline\endfirsthead
\caption{Rep�res de r�f�rence (suite)}\\
\hline
Mots cl�s en Fran\c{c}ais & Mots cl�s en Anglais\\
\hline\endhead
\hline\multicolumn{2}{|r|}{� suivre ...}\\\hline\endfoot
\hline\endlastfoot

geocentrique &
  geocentric \\

inertiel &
  inertial \\

orbital-TNW &
  TNW-orbital \\

orbital-QSW &
  QSW-orbital \\

topocentrique &
  topocentric \\

utilisateur &
  user\\

\hline
\end{longtable}\end{center}

\begin{center}\begin{longtable}{|c|c|}
\caption{Astres et cibles connus}\\
\hline
Mots cl�s en Fran\c{c}ais & Mots cl�s en Anglais\\
\hline\endfirsthead
\caption{Astres et cibles connus (suite)}\\
\hline
Mots cl�s en Fran\c{c}ais & Mots cl�s en Anglais\\
\hline\endhead
\hline\multicolumn{2}{|r|}{� suivre ...}\\\hline\endfoot
\hline\endlastfoot

canopus &
   canopus \\

canopus-sans-eclipse 
&
   eclipse-free-canopus \\

corps-central &
   central-body \\

corps-central-soleil &
   central-body-sun \\

devant &
   along-track \\

direction &
   direction \\

direction-sans-eclipse &
   eclipse-free-direction \\

nadir &
   nadir \\

lune &
   moon \\

lune-sans-eclipse &
   eclipse-free-moon \\

moment &
   momentum \\

polaris &
   polaris \\

polaris-sans-eclipse 
&
   eclipse-free-polaris \\

position &
   position \\

position-sans-eclipse &
   eclipse-free-position \\

pseudo-soleil &
   pseudo-sun \\

soleil &
   sun \\

soleil-sans-eclipse &
   eclipse-free-sun \\

station &
   station \\

terre-soleil &
   earth-sun \\

vitesse &
   velocity \\

vitesse-sol-apparente &
   apparent-ground-velocity \\

\hline
\end{longtable}\end{center}

\clearpage\section{Lexique Anglais-Fran�ais des mots cl�s du fichier
   Senseurs}\label{sec:lexique-en}

\begin{center}\begin{longtable}{|c|c|}
\caption{Mots-cl�s du fichier senseurs}\\
\hline
Mots cl�s en Anglais & Mots cl�s en Fran\c{c}ais\\
\hline\endfirsthead
\caption{Mots-cl�s du fichier senseurs (suite)}\\
\hline
Mots cl�s en Anglais & Mots cl�s en Fran\c{c}ais\\
\hline\endhead
\hline\multicolumn{2}{|r|}{� suivre ...}\\\hline\endfoot
\hline\endlastfoot

accuracy &
   precision \\

altitude &
   altitude \\

angle &
   angle \\

angle\_3dB &
   angle\_3dB \\

angle\_3dB\_x &
   angle\_3dB\_x \\

angle\_3dB\_y &
   angle\_3dB\_y \\

axis &
   axe \\

central\_body\_field\_of\_inhibition  
&
   champ\_d\_inhibition\_corps\_central \\

cone &
   cone \\

except &
   sauf \\

field\_of\_view &
   champ\_de\_vue \\

frame &
   repere \\

hygrometry &
   hygrometrie \\

i &
   i \\

inter &
   inter \\

j &
   j \\

k &
   k \\

kind &
   genre \\

longitude &
   longitude \\

latitude &
   latitude \\

mask &
   masque \\

maximum &
  maximum \\

moon\_eclipse\_margin &
   marge\_eclipse\_lune
\footnote{Pour plus de pr�cisions: se reporter au \S~\ref{sec:schemaInhibition} } 
\\

moon\_field\_of\_inhibition &
   champ\_d\_inhibition\_lune \\

moon\_phase\_threshold &
   seuil\_phase\_lune 
\footnote{Pour plus de pr�cisions: se reporter au \S~\ref{sec:schemaInhibition} } 
\\

normal\_reference &
   normale\_reference \\

observed &
   observe \\

of &
   de \\

origin &
   origine \\

pressure &
   pression \\

reference &
   reference \\

rotation &
   rotation \\

sample &
  echantillon \\

sensitive\_axis &
   axe\_sensible \\

spread &
   balayage \\

sun\_eclipse\_margin &
   marge\_eclipse\_soleil
\footnote{Pour plus de pr�cisions: se reporter au \S~\ref{sec:schemaInhibition} } 
\\

sun\_field\_of\_inhibition &
   champ\_d\_inhibition\_soleil \\

target &
   cible \\

temperature &
  temperature \\

type &
   type \\

union &
   union \\

v\_base &
   v\_base \\

v\_base\_1 &
   v\_base\_1 \\

v\_base\_2 &
   v\_base\_2 \\

v\_image &
   v\_image \\

v\_image\_1 &
   v\_image\_1 \\

v\_image\_2 &
   v\_image\_2 \\

wedging\_axis &
  axe\_calage \\

zero\_angle &
   angle\_zero \\

zero\_reference &
   reference\_zero \\

\hline
\end{longtable}\end{center}


\begin{center}\begin{longtable}{|c|c|}
\caption{Types de senseurs reconnus}\\
\hline
Mots cl�s en Anglais & Mots cl�s en Fran\c{c}ais\\
\hline\endfirsthead
\caption{Types de senseurs reconnus (suite)}\\
\hline
Mots cl�s en Anglais & Mots cl�s en Fran\c{c}ais\\
\hline\endhead
\hline\multicolumn{2}{|r|}{� suivre ...}\\\hline\endfoot
\hline\endlastfoot

cardan &
   cardan \\

cartesian &
   cartesien \\

declination &
   declinaison \\

dihedral &
   diedre \\

earth &
   terre \\

gauss\_gain &
   gain\_gauss \\

integrating\_gyro &
   gyro\_integrateur \\

limb &
   limbe \\

kinematic &
   cinematique \\

plane\_vector &
   plan\_vecteur \\

right\_ascension &
   ascension\_droite \\

sampled\_1D\_gain &
   gain\_echantillonne\_1D \\

square\_cardinal\_sine\_gain &
   gain\_sinus\_cardinal\_2 \\

vector &
   vecteur \\

xy\_cardinal\_sine\_gain &
   gain\_sinus\_cardinal\_xy \\

\hline
\end{longtable}\end{center}

\begin{center}\begin{longtable}{|c|c|}
\caption{Types de senseurs cardans}\\
\hline
Mots cl�s en Anglais & Mots cl�s en Fran\c{c}ais\\
\hline\endfirsthead
\caption{Types de senseurs cardans (suite)}\\
\hline
Mots cl�s en Anglais & Mots cl�s en Fran\c{c}ais\\
\hline\endhead
\hline\multicolumn{2}{|r|}{� suivre ...}\\\hline\endfoot
\hline\endlastfoot

PRY-pitch &
   TRL-tangage \\

PRY-roll &
   TRL-roulis \\

PRY-yaw &
   TRL-lacet \\

PYR-pitch &
   TLR-tangage \\

PYR-roll &
   TLR-roulis \\

PYR-yaw &
  TLR-lacet \\

RPY-pitch &
   RTL-tangage \\

RPY-roll &
   RTL-roulis \\

RPY-yaw &
   RTL-lacet \\

RYP-pitch &
   RLT-tangage \\

RYP-roll &
    RLT-roulis \\

RYP-yaw &
   RLT-lacet \\

YPR-pitch &
   LTR-tangage \\

YPR-roll &
   LTR-roulis \\

YPR-yaw &
   LTR-lacet \\

YRP-pitch &
   LRT-tangage \\

YRP-roll &
   LRT-roulis \\

YRP-yaw &
   LRT-lacet \\

\hline
\end{longtable}\end{center}

\begin{center}\begin{longtable}{|c|c|}
\caption{Rep�res de r�f�rence}\\
\hline
Mots cl�s en Anglais & Mots cl�s en Fran\c{c}ais\\
\hline\endfirsthead
\caption{Rep�res de r�f�rence (suite)}\\
\hline
Mots cl�s en Anglais & Mots cl�s en Fran\c{c}ais\\
\hline\endhead
\hline\multicolumn{2}{|r|}{� suivre ...}\\\hline\endfoot
\hline\endlastfoot

geocentric &
   geocentrique \\

inertial &
   inertiel \\

QSW-orbital &
   orbital-QSW \\

TNW-orbital &
   orbital-TNW \\

topocentric &
   topocentrique \\

user &
   utilisateur \\

\hline
\end{longtable}\end{center}

\begin{center}\begin{longtable}{|c|c|}
\caption{Astres et cibles connus}\\
\hline
Mots cl�s en Anglais & Mots cl�s en Fran\c{c}ais\\
\hline\endfirsthead
\caption{Astres et cibles connus (suite)}\\
\hline
Mots cl�s en Anglais & Mots cl�s en Fran\c{c}ais\\
\hline\endhead
\hline\multicolumn{2}{|r|}{� suivre ...}\\\hline\endfoot
\hline\endlastfoot

along-track &
   devant \\

apparent-ground-velocity &
   vitesse-sol-apparente \\

canopus &
   canopus \\

central-body &
   corps-central \\

central-body-sun &
   corps-central-soleil \\

direction &
   direction \\

earth-sun &
   terre-soleil \\

eclipse-free-canopus 
&
   canopus-sans-eclipse \\

eclipse-free-direction &
   direction-sans-eclipse \\

eclipse-free-moon &
   lune-sans-eclipse \\

eclipse-free-polaris 
&
   polaris-sans-eclipse \\

eclipse-free-position &
   position-sans-eclipse \\

eclipse-free-sun &
   soleil-sans-eclipse \\

momentum &
   moment \\

moon &
   lune \\

nadir &
   nadir \\

polaris &
   polaris \\

position &
   position \\

pseudo-sun &
   pseudo-soleil \\

station &
   station \\

sun &
   soleil \\

velocity &
   vitesse \\

\hline
\end{longtable}\end{center}

\clearpage\section{D�finitions des rep�res utilis�s}\label{sec:def-reperes}

Marmottes utilise principalement trois grandes cat�gorie de rep�re.\\
\\
Le rep�re inertiel est le rep�re dont l'origine est au centre du corps
attracteur et les axes sont fixes dans l'espace (gamma50 CNES, J2000,
...). Toutes les positions et vitesses pass�es en argument � Marmottes
sont exprim�es dans ce rep�re. L'attitude est la rotation qui,
appliqu�e aux coordonn�es d'un vecteur exprim� dans ce rep�re, donne les
coordonn�es, de ce m�me vecteur, exprim�es en rep�re satellite.\\
\\
Le rep�re satellite est le rep�re dont on cherche � d�terminer
l'orientation par rapport au rep�re inertiel. Il est d�fini par le
constructeur et correspond au coprs du satellite.\\
\\
Les rep�res senseurs sont les rep�res propres aux �quipements de
mesure de l'attitude (typiquement la t�te optique des senseurs ou le
bo�tier des gyrom�tres). Ce rep�re est cal�, par construction, par
rapport au rep�re satellite. C'est ce rep�re qui doit �tre d�fini pour
chaque senseur dans le fichier des senseurs. Tous les vecteurs de
d�finition des axes de vis�e, de mesure, de champ de vue des senseurs
sont exprim�s dans ce rep�re dans le fichier senseurs.\\
\\
Outre ces rep�res g�n�raux, les capteurs de Cardan utilisent des
rep�res sp�cifiques.
Les capteurs d'angles de Cardan mesurent les angles de rotation
successives permettant de passer d'un rep�re de r�f�rence au rep�re
satellite. Plusieurs rep�re de r�f�rence sont pr�d�finis et peuvent
�tre sp�cifi�s dans le fichier senseurs. Ces rep�res sont d�finis de
la fa�on suivante, par rapport au rep�re inertiel (dans ces
d�finitions, $\vec{P}$ est le vecteur position du satellite, compt� du
centre du corps attracteur vers le satellite et $\vec{V}$ est le
vecteur vitesse du satellite).

\begin{description}

\item[] rep�re g�ocentrique \\
    Ce rep�re d�pend de la position du satellite et
    tourne � la fr�quence orbitale. \\
    Ce rep�re est d�fini par :
\begin{itemize}
    \item $\vec{Z}$ est dirig� du satellite vers le centre du corps
    attracteur ($\vec{Z} = -\vec{P} / ||\vec{P}||$),
    \item $\vec{Y}$ est port� par l'oppos� du moment cin�tique
    ($\vec{Y} = -\vec{P} \wedge \vec{V} / ||\vec{P} \wedge \vec{V}||$),
    \item $\vec{X}$ compl�te le tri�dre ($\vec{X} = \vec{Y} \wedge \vec{Z}$).
\end{itemize}

\item[] rep�re QSW \\
    Ce rep�re d�pend de la position du satellite et
    tourne � la fr�quence orbitale. \\
    Ce rep�re est d�fini par :
\begin{itemize}
    \item $\vec{X}$ pointe vers l'oppos� du centre du corps attracteur
    ($\vec{X} = -\vec{P} \wedge ||\vec{P}||$),
    \item $\vec{Z}$ est port� par le moment orbital ($\vec{Z} =
    \vec{P} \wedge \vec{V} / ||\vec{P} \wedge \vec{V}||$),
    \item $\vec{Y}$ compl�te le tri�dre ($\vec{Y} = \vec{Z} \wedge
    \vec{X}$).
\end{itemize}

\item[] rep�re topocentrique \\
    Ce rep�re d�pend de la position du satellite et
    tourne � la fr�quence orbitale. \\
    Ce rep�re est d�fini par :
\begin{itemize}
    \item $\vec{Z}$ pointe vers le centre du corps attracteur
    ($\vec{Z} = -\vec{P} / ||\vec{P}||$),
    \item $\vec{Y}$ pointe vers l'Est, ses coordonn�es sont ($-P_y /
    \sqrt{P_x^2 + P_y^2}, P_x / \sqrt{P_x^2 + P_y^2}, 0$) en rep�re
    inertiel,
    \item $\vec{X}$ compl�te le tri�dre ($\vec{X} = \vec{Y} \wedge
    \vec{Z}$).
\end{itemize}

\item[] rep�re inertiel \\
    Ce rep�re ne d�pend de rien et est fixe. \\
    Ce rep�re est le rep�re de d�finition. Ses axes sont donc les axes
    canoniques ($\vec{X}$ (1,0,0),$\vec{Y}$ (0,1,0),$\vec{Z}$ (0,0,1)).

\item[] rep�re TNW \\
    Ce rep�re d�pend de la position du satellite et
    tourne � la fr�quence orbitale. \\
    Ce rep�re est d�fini par :
\begin{itemize}
    \item $\vec{X}$ est port� par la vitesse ($\vec{X} = \vec{V} /
    ||\vec{V}||$),
    \item $\vec{Z}$ est port� par le moment orbital ($\vec{Z} =
    \vec{P} \wedge \vec{V} / ||\vec{P} \wedge \vec{V}||$),
    \item $\vec{Y}$ compl�te le tri�dre ($\vec{Y} = \vec{Z} \wedge
    \vec{X}$).
\end{itemize}
     
\item[] rep�re utilisateur \\
    Ce rep�re est enti�rement param�tr� par l'utilisateur � l'aide de
    la fonction \fonc{MarmottesModifieReference}. Il s'agit typiquement de
    l'attitude retourn�e par un appel pr�alable �
    \fonc{MarmottesAttitude}. Ceci permet alors de consid�rer les mesures des
    senseurs de Cardan comme les \emph{�carts} (ou les erreurs de
    pilotage) par rapport � cette attitude de r�f�rence.
\end{description}

