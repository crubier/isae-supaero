% -*- mode: latex; tex-main-file: "marmottes.tex" -*-
\subsection{marmottes-config}\label{sec:MarmottesConfig}

\subsubsection*{description g�n�rale}\label{sec:MarmottesConfig-desc}

L'utilitaire \texttt{marmottes-config} permet de d�terminer les options
n�cessaires � la compilation d'applicatifs ou de biblioth�ques
s'appuyant sur \bibliotheque{marmottes}. Il est principalement destin� �
�tre utilis� dans les r�gles des fichiers de directives du type
\texttt{Makefile}.

\subsubsection*{ligne de commande et options}\label{sec:MarmottesConfig-lignecmde}

La ligne de commande a la forme suivante~:

\begin{verbatim}
   marmottes-config [--cppflags | --ldflags | --libs | --version]
\end{verbatim}

Si aucune des options \texttt{-{}-cppflags}, \texttt{-{}-ldflags},
\texttt{-{}-libs} ou \texttt{-{}-version} n'est utilis�e, l'utilitaire
indique les options disponibles et s'arr�te avec un code d'erreur.

\subsubsection*{descriptions des sorties}\label{sec:MarmottesConfig-sorties}

Les sorties de l'utilitaire d�pendent des options s�lectionn�es dans
la ligne de commande.

L'option \texttt{-{}-cppflags} permet d'obtenir sur la sortie standard les
options de compilation, comme dans l'exemple suivant~:
\begin{verbatim}
  (lehrin) luc% marmottes-config --cppflags
   -I/home/luc/include
  (lehrin) luc%
\end{verbatim}

L'option \texttt{-{}-ldflags} permet d'obtenir sur la sortie standard les
options d'�dition de liens, comme dans l'exemple suivant~:
\begin{verbatim}
  (lehrin) luc% marmottes-config --ldflags
   -L/home/luc/lib
  (lehrin) luc%
\end{verbatim}

L'option \texttt{-{}-libs} permet d'obtenir sur la sortie standard les
biblioth�ques n�cessaires � l'�dition de liens, comme dans l'exemple
suivant~:
\begin{verbatim}
  (lehrin) luc% marmottes-config --libs
   -lmarmottes -lcantor -lclub -lxerces 
  (lehrin) luc%
\end{verbatim}

L'option \texttt{-{}-version} permet d'obtenir sur la sortie standard
le num�ro de version de la biblioth�que, comme dans l'exemple
suivant~:
\begin{verbatim}
  (lehrin) luc% marmottes-config --version
   5.5
  (lehrin) luc%
\end{verbatim}

\subsubsection*{conseils d'utilisation}\label{sec:MarmottesConfig-conseils}

L'utilisation classique de \texttt{-{}-cppflags} est dans une r�gle de
compilation de fichier \texttt{Makefile} du type~:

\begin{verbatim}
  client.o : client.cc
       $(CXX) `marmottes-config --cppflags` $(CPPFLAGS) $(CXXFLAGS) -c client.cc
\end{verbatim}

L'utilisation classique de \texttt{-{}-ldflags} et \texttt{-{}-libs}
est dans une r�gle d'�dition de liens de fichier \texttt{Makefile} du
type~:

\begin{verbatim}
  client : client.o
       $(CXX) -o $@ client.o \
              `marmottes-config --ldflags` $(LDFLAGS) \
              `marmottes-config --libs`    $(LIBS)
\end{verbatim}

Il est possible de combiner les options \texttt{-{}-cppflags},
\texttt{-{}-ldflags} et \texttt{-{}-libs} dans une r�gle de fichier
\texttt{Makefile} du type~:

\begin{verbatim}
  client : client.cc
       $(CXX) -o $@ `marmottes-config --cppflags --ldflags --libs` \
              $(CPPFLAGS) $(CXXFLAGS) $(LDFLAGS) $(LIBS) client.cc
\end{verbatim}
