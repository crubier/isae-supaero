% -*- mode: latex; tex-main-file: "marmottes-utilisateur.tex" -*-
% $Id: SpinAtt.tex,v 1.5 2003/07/09 09:24:57 marmottes Exp $
\subsection{classe SpinAtt}\label{sec:SpinAtt}

\subsubsection*{description}\label{sec:SpinAtt-desc}

Cette classe est un conteneur permettant de m�moriser un couple
attitude et spin. Elle est utilis�e par la classe ResolveurAttitude
pour stocker toutes les solutions trouv�es.
\subsubsection*{interface publique}\label{sec:SpinAtt-int}
\begin{verbatim}
#include "marmottes/SpinAtt.h"
\end{verbatim}
\begin{tableFonctionsFixe}{SpinAtt : m�thodes publiques}
{\label{tab:SpinAtt-met-pub}}
{construit une instance � partir d'une \argument{attitude}}

\signature{\fonc{SpinAtt} ()}
          {}&

constructeur par d�faut\\

\signature{\fonc{SpinAtt}}
          {(const RotDBL\& \argument{attitude},\\
            const VecDBL\& \argument{spin})
          }&

construit une instance � partir d'une \argument{attitude} et d'un
\argument{spin}\\

\hline

\signature{\fonc{SpinAtt} (const SpinAtt\& \argument{sa})}
          {}&

constructeur par copie\\

\signature{SpinAtt\& \fonc{operator =} (const SpinAtt\& \argument{s})}
          {}&

affectation\\

\signature{\fonc{\~{}SpinAtt()}}{}
          {}&

destructeur
\\

\hline

\signature{const RotDBL\& \fonc{attitude} () const}
          {}&

retourne le spin m�moris�\\

\signature{const VecDBL\& \fonc{spin} () const}
          {}&

retourne l'attitude m�moris�e\\

\end{tableFonctionsFixe}
\subsubsection*{exemple d'utilisation}

\begin{verbatim}
#include "marmottes/SpinAtt.h"

void  Marmottes::attitude (double date,
                           const VecDBL& position, const VecDBL& vitesse,
                           double m1, double m2, double m3,
                           RotDBL *attitude, VecDBL *spin)
  throw (CantorErreurs, MarmottesErreurs)
{ // calcul d'une attitude donn�e par trois consignes
  try
  {

    ...

    // r�cup�ration de la "meilleure" solution
    SpinAtt sol = solveur_.selection ();
    *attitude   = sol.attitude ();
    *spin       = sol.spin     ();

    ...
  }

  catch (...)
  {
    etatExtrapole_ = etatResolu_;
    throw;
  }
}
\end{verbatim}

\subsubsection*{conseils d'utilisation sp�cifiques}
\label{sec:SpinAtt-conseils}
Cette classe est un simple conteneur conservant des copies des
�l�ments avec lesquels l'instance est construite, elle ne pr�sente
aucune difficult� particuli�re d'utilisation.

\subsubsection*{implantation}\label{sec:SpinAtt-impl}
Les attributs priv�s sont d�crits sommairement dans la
table~\ref{tab:SpinAtt-att-priv}, il n'y a pas d'attribut prot�g�.
\begin{tableAttributsFixe}{attributs priv�s de la classe SpinAtt}
{\label{tab:SpinAtt-att-priv}}
{attitude m�moris�e}

attitude\_ & RotDBL  & attitude m�moris�e\\

spin\_ & VecDBL  & spin m�moris�\\

\end{tableAttributsFixe}
