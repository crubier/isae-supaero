% -*- mode: latex; tex-main-file: "marmottes-utilisateur.tex" -*-
% $Id: FamilleFixe.tex,v 1.4 2002/02/27 13:36:02 marmottes Exp $
\subsection{classe FamilleFixe}\label{sec:FamilleFixe}

\subsubsection*{description}\label{sec:FamilleFixe-desc}

Cette classe implante les cas particuliers du mod�le analytique � un
degr� de libert� respectant deux consignes g�om�triques pour lesquels
les vecteurs $\vec{v}_1$ ou $\vec{v}_2$ sont fix�s.

\subsubsection*{interface publique}\label{sec:FamilleFixe-int}
\begin{verbatim}
#include "marmottes/FamilleFixe.h"
\end{verbatim}
\begin{tableFonctionsFixe}{FamilleFixe : m�thodes publiques}
{\label{tab:FamilleFixe-met-pub}}
{m�thode virtuelle pure de la classe FamilleAbstraite, }

\signature{\fonc{FamilleFixe}()} 
          {}& 

initialise une instance par d�faut inutilisable sans r�affectation\\

\hline

\signature{\fonc{FamilleFixe}}
          {(const Intervalle \argument{plages}, \\
           const VecVD1 \argument{u1}, const VecVD1 \argument{u2},\\
           const VecVD1 \argument{v1}, const VecVD1 \argument{ref},\\
           const VecVD1 \argument{axe})} & 

construit une FamilleFixe � partir d'un intervalle plages, des
vecteurs $\vec{u}_1$ et $\vec{u}_2$ exprim�s dans le rep�re inertiel,
du vecteur $\vec{v}_1$ exprim� dans le rep�re canonique (d�fini dans
la ModeleGeom) et des r�f�rence et axe qui d�finissent le param�tre
libre sur le secteur de consigne consid�r�\\

\hline

\signature{\fonc{FamilleFixe}(const FamilleFixe\& \argument{f})} 
           {}& 

constructeur par copie\\

\signature{FamilleFixe\&  \fonc{operator =}(const FamilleFixe\&
\argument{f})} 
          {}&

affectation\\

\signature{FamilleAbstraite *  \fonc{copie}() const} 
          {}& 

op�rateur de copie\\

\signature{\fonc{\~{}FamilleFixe}()} 
          {}&

destructeur\\

\hline

\signature{RotVD1  \fonc{inertielCanonique}}{(const ValeurDerivee1\&
\argument{t}) const} & 

m�thode virtuelle pure de la classe FamilleAbstraite, red�finie ici et
qui retourne le quaternion de passage du rep�re inertiel au rep�re canonique de travail d�fini dans ModeleGeom\\

\end{tableFonctionsFixe}
\subsubsection*{exemple d'utilisation}
Les exemples vus pour les classes FamilleAlignementMoins,
FamilleAlignementPlus, directement extraits du code de la biblioth�que
montrent comment, dans la classe ModeleGeom, on a cr�� un vecteur (au
sens de la \bibliotheque{stl}~\ref{ref:stl}) de FamilleFixe.

\subsubsection*{conseils d'utilisation
sp�cifiques}\label{sec:FamilleFixe-conseils}
Par rapport aux anciennes versions de \bibliotheque{marmottes}, cette
famille est limit�e par les champs de vue et par les domaines de
validit� du param�tre de consigne. Ceci apporte un gain dans le temps
de calcul.

\subsubsection*{implantation}\label{sec:FamilleFixe-impl}
Les attributs priv�s sont d�crits sommairement dans la
table~\ref{tab:FamilleFixe-att-priv}, il n'y a pas d'attribut prot�g�.
\begin{tableAttributsFixe}{attributs priv�s de la classe FamilleFixe}
{\label{tab:FamilleFixe-att-priv}}
{coordonn�es en rep�re inertiel du vecteur fixe}

axe\_ & VecVD1 & coordonn�es en rep�re canonique de l'axe du c�ne de
consigne\\

r\_ & RotVD1  & rotation constante qui permet d'amener le vecteur fixe
� sa place dans le rep�re canonique\\

\end{tableAttributsFixe}
