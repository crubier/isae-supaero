% -*- mode: latex; tex-main-file: "cantor.tex" -*-
% $Id: fonctions-C.tex,v 1.2 2001/06/05 16:13:14 cantor Exp $
\subsection{fonctions \langage{c}}\label{sec:fonctions-c}
\begin{tableFonctionsFixe}{fonctions C}
{\label{tab:fonctions-c}}
{construit la rotation \argument{q} r�sultant de la composition}

\signature{int \fonc{RotAxeAngle}}{(double \argument{q} [4], double
\argument{axe} [3], double \argument{angle},\\char*
\argument{message}, int \argument{lgMaxMessage})} & construit la
rotation \argument{q} � partir de l'\argument{axe} et de
l'\argument{angle}, retourne un code non nul en cas de probl�me, et
initialise le \argument{message} d'erreur\\

\signature{int \fonc{RotU1U2V1V2}}{(double \argument{q} [4], double
\argument{u1} [3], double \argument{u2} [3],\\double \argument{v1}
[3], double \argument{v2} [3],\\char* \argument{message}, int
\argument{lgMaxMessage})} & construit la rotation \argument{q} �
partir des deux vecteurs \argument{u1} et \argument{u2} et de leurs
images \argument{v1} et \argument{v2}, retourne un code non nul en cas
de probl�me, et initialise le \argument{message} d'erreur\\

\signature{int \fonc{RotU1V1}}{(double \argument{q} [4], double
\argument{u1} [3], double \argument{v1} [3],\\char*
\argument{message}, int \argument{lgMaxMessage})} & construit la
rotation \argument{q} � partir du vecteur \argument{u1} et de son
image \argument{v1} (une solution est choisie arbitrairement dans
l'infinit� possible), retourne un code non nul en cas de probl�me, et
initialise le \argument{message} d'erreur\\

\signature{int \fonc{RotMatrice}}{(double \argument{q} [4], double
\argument{m} [3][3], double \argument{seuil},\\char*
\argument{message}, int \argument{lgMaxMessage})} & construit la
rotation \argument{q} � partir de la matrice \argument{m} en
corrigeant �ventuellement sa non-orthogonalit� d'au plus
\argument{seuil} (au sens de la norme de \textsc{Frobenius}), retourne
un code non nul en cas de probl�me, et initialise le
\argument{message} d'erreur\\

\signature{int \fonc{RotTroisAngles}}{(double \argument{q} [4],
CantorAxesRotation \argument{ordre},\\double \argument{alpha1}, double
\argument{alpha2}, double \argument{alpha3},\\char *\argument{message}, int
\argument{lgMaxMessage})} & construit la rotation \argument{q} �
partir de trois rotations �l�mentaires dans l'\argument{ordre}
sp�cifi�, retourne un code non nul en cas de probl�me, et initialise
le \argument{message} d'erreur\\

\signature{void \fonc{RotInverse}}{(double \argument{q} [4], double
\argument{qInitiale} [4])} & construit la rotation \argument{q}
inverse de \argument{qInitiale}, en se contentant d'inverser un
�l�ment du quaternion (cette op�ration est donc tr�s peu co�teuse en
temps de calcul)\\

\signature{void \fonc{RotComposee}}{(double \argument{q} [4], double
\argument{q1} [4], double \argument{q2} [4])} & construit la rotation
\argument{q} r�sultant de la composition \argument{q2} $\circ$
\argument{q1}\\

\hline

\signature{void \fonc{AxeRot}(double \argument{q} [4], double
\argument{axe} [3])}{} & extrait l'axe de la rotation
\argument{q} et le met dans le tableau \argument{axe}\\

\signature{void \fonc{AngleRot}(double \argument{q} [4], double*
\argument{pAngle})}{} & extrait l'angle de la rotation
\argument{q} et le met dans la variable point�e par \argument{pAngle}\\

\signature{void \fonc{AxeAngleRot}}{(double \argument{q} [4], double
\argument{axe} [3], double* \argument{pAngle})} & extrait l'axe et
l'angle de la rotation \argument{q} et les met dans le tableau
\argument{axe} et dans la variable point�e par \argument{pAngle}\\

\signature{void \fonc{MatriceRot}(double \argument{q} [4], double
\argument{m} [3][3])}{} & extrait la matrice de la rotation
\argument{q} et la met dans le tableau \argument{m}\\

\signature{int \fonc{TroisAnglesRot}}{(double \argument{q} [4],
CantorAxesRotation \argument{ordre},\\double *\argument{pAlpha1}, double
*\argument{pAlpha2}, double *\argument{pAlpha3},\\char
*\argument{message}, int \argument{lgMaxMessage})} & extrait de la
rotation \argument{q} les angles des rotations �l�mentaires dans
l'\argument{ordre} sp�cifi�, retourne un code non nul en cas de
probl�me, et initialise le \argument{message} d'erreur\\

\hline

\signature{void \fonc{AppliqueRot}}{(double \argument{q} [4], double
\argument{u} [3], double \argument{uPrime} [3])} & applique la
rotation \argument{q} au vecteur \argument{u} et met l'image dans le
tableau \argument{uPrime}\\

\signature{void \fonc{AppliqueRotInverse}}{(double \argument{q} [4],
double \argument{uPrime} [3], double \argument{u} [3])} & applique
l'inverse de la rotation \argument{q} au \argument{uPrime} et met
l'image r�ciproque dans le tableau \argument{u}\\

\end{tableFonctionsFixe}
