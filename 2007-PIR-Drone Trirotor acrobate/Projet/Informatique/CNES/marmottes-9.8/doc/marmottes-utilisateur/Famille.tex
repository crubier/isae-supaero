% -*- mode: latex; tex-main-file: "marmottes-utilisateur.tex" -*-
% $Id: Famille.tex,v 1.2 1999/09/17 15:29:35 luc Exp $
\subsection{classe Famille}\label{sec:Famille}

\subsubsection*{description}\label{sec:Famille-desc}

Cette classe ne sert que d'interface � la classe FamilleAbstraite. Le
langage \langage{c++} ne permettant de g�rer des instances d�rivant
d'une classe abstraite qu'� l'aide de pointeurs, cette classe
encapsule de tels pointeurs et s'occupe de la gestion des copies et
des d�sallocations m�moires de fa�on � pr�senter ces instances comme
s'il s'agissait d'objets simples.

Elle est utilis�e directement par la classe ModeleGeom qui r�soud le
mod�le � un degr� de libert� respectant deux consignes dans le cas de
deux senseurs g�om�triques.
\subsubsection*{interface publique}\label{sec:Famille-int}
\begin{verbatim}
#include "marmottes/Famille.h"
\end{verbatim}
\begin{tableFonctionsFixe}{Famille : m�thodes publiques}
{\label{tab:Famille-met-pub}}
{initialise une instance par d�faut inutilisable}

\signature{\fonc{Famille} ()} 
          {}&
initialise une instance par d�faut inutilisable sans r�affectation (pointeur nul)\\

\hline

\signature{\fonc{Famille} (const FamilleAbstraite* \argument{f})} 
          {}&
m�thode qui permet de construire une Famille � partir d'un pointeur de
FamilleAbstraite (copie profonde de l'objet point�)\\

\hline

\signature{\fonc{Famille} (const Famille\& \argument{f})} 
          {}&
constructeur par copie \\

\signature{Famille\&  \fonc{operator =} (const Famille\& \argument{f})} 
          {}& 
affectation \\

\signature{\fonc{\~{}Famille} ()} 
          {}& 
destructeur, d�truit le pointeur\\

\hline

\signature{RotVD1  \fonc{inertielCanonique}}{(const ValeurDerivee1\&
\argument{t}) const} & m�thode qui retourne le quaternion de passage
du rep�re inertiel au rep�re canonique de travail d�fini dans ModeleGeom \\

\end{tableFonctionsFixe}
\subsubsection*{exemple d'utilisation}
L'exemple suivant, directement extrait du code de la biblioth�que
montre comment on a cr�� un vecteur (au sens de la \bibliotheque{stl}~\ref{ref:stl})
de Famille.

\begin{verbatim}
#include "marmottes/Famille.h"

{
  Intervalle i = Intervalle (-M_PI , M_PI);
  // cr�ation d'une Famille
  Famille f (new FamilleFixe (i, u1, u2, 
                             (-canSat_) (VecDBLVD1 (v1a)),
                             (-canSat_) (VecDBLVD1 (v1a))));
  // table_ est de type vector 
  table_.push_back (f);
  nombreFamilles_++;
}
\end{verbatim}

\subsubsection*{conseils d'utilisation
sp�cifiques}\label{sec:Famille-conseils}
Cette classe n'a �t� impl�ment�e que pour des raisons
informatiques. En effet, elle permet de g�rer �l�gamment la notion de
pointeur de FamilleAbstraite (copie, destruction ...).

\subsubsection*{implantation}\label{sec:Famille-impl}
Les attributs priv�s sont d�crits sommairement dans la
table~\ref{tab:Famille-att-priv}, il n'y a pas d'attribut prot�g�.
\begin{tableAttributsFixe}{attributs priv�s de la classe Famille}
{\label{tab:Famille-att-priv}}
{pointeur sur un objet de type FamilleAbstraite}

ptrFamille\_ & FamilleAbstraite *  & pointeur sur un objet de type FamilleAbstraite \\

\end{tableAttributsFixe}
