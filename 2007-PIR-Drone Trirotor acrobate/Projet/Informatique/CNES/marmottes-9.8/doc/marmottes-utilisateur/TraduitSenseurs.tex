% -*- mode: latex; tex-main-file: "marmottes.tex" -*-
\subsection{traduitSenseurs}\label{sec:TraduitSenseurs}

\subsubsection*{description g�n�rale}\label{sec:TraduitSenseurs-desc}

L'utilitaire \texttt{traduitSenseurs} permet de traduire des fichiers
de capteurs de fran�ais � anglais et r�ciproquement.

\subsubsection*{ligne de commande et options}\label{sec:TraduitSenseurs-lignecmde}

La ligne de commande a la forme suivante~:

\begin{verbatim}
   traduitSenseurs fichier
\end{verbatim}

Le seul argument est le nom du fichier de base des capteurs (ce
fichier peut en inclure d'autres). Tous les noms de fichiers doivent
�tre de la forme \texttt{nom.en} ou \texttt{nom.fr}, le suffixe
sp�cifiant la langue.

\subsubsection*{descriptions des sorties}\label{sec:TraduitSenseurs-sorties}

L'utilitaire n'affiche rien sur sa sortie standard, il cr�e
directement les fichiers traduits dans le m�me r�pertoire que les
fichiers d'origine, en se basant sur le suffixe pour d�terminer si la
traduction est du fran�ais vers l'anglais ou de l'anglais vers le fran�ais.

Seuls les mot-clefs et les noms de fichiers inclus sont traduits, les
commentaires sont pr�serv�s. Lorsqu'un fichier r�f�rence d'autres
fichiers, ceux-ci sont traduits �galement.

\subsubsection*{conseils d'utilisation}\label{sec:TraduitSenseurs-conseils}

Les fichiers de capteurs sont souvent �crits � la main, et les
r�dacteurs apportent souvent une certain attention � l'indentation des
structures pour am�liorer la lisibilit�. Lors de la traduction, tous les
blancs sont pr�serv�s, mais comme la taille des mots-clefs varie entre
anglais et fran�ais, ceci perturbe l'indentation. Si l'on a besoin de
maintenir les fichiers dans les deux langues, il est recommand� de
faire les mises � jour manuelles toujours � partir de la m�me langue
et de g�n�rer automatiquement les fichiers de l'autre langue. En effet
si l'on travaille alternativement sur les deux langues, on finit par
obtenir des portions peu lisibles dans tous les fichiers.
