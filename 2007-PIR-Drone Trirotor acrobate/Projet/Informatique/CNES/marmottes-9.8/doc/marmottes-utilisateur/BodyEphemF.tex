% $Id: BodyEphemF.tex,v 1.1 2002/09/09 12:59:59 marmottes Exp $
\subsection{classe BodyEphemF}\label{sec:BodyEphemF}

\subsubsection*{description}\label{sec:BodyEphemF-desc}

Cette classe d�riv�e de la classe BodyEphem implante l'interface
d'acc�s, pour l'impl�mentation en \langage{fortran} par l'utilisateur,
du calcul du temps sid�ral et des �ph�m�rides du Soleil,
de la Lune et de la Terre, par rapport au corps central.\\
Elle permet aussi l'acc�s � des grandeurs physiques du corps central
(rayon �quatorial, aplatissement et vitesse de rotation).


\subsubsection*{interface publique}\label{sec:BodyEphemF-int}
\begin{verbatim}
#include "marmottes/BodyEphemF.h"
\end{verbatim}
\begin{tableFonctionsFixe}{BodyEphemF : m�thodes publiques}
{\label{tab:BodyEphemF-met-pub}}
{xxxxxxxxxxxxxxxxxxxxxxxxxxxxxxxxxxxxxxxxxxxxxxxxxxxxxxxxxxxxxx}


\signature{typedef double  \fonc{TypeFuncTsid}}{(double *, double *)} & 
d�fini la signature, pour le \langage{fortran}, de la fonction de calcul du
temps sid�ral \\

\signature{typedef void  \fonc{TypeFuncPos}}{(double *, double [3])} &
d�fini la signature pour le \langage{fortran} des fonctions de calcul de
position des corps (Soleil, Lune et Terre) par rapport au corps central\\

\hline

\signature{\fonc{BodyEphemF} ()}{}  & 
initialise une instance par d�faut \\

\signature{\fonc{BodyEphemF} (const BodyEphemF\& \argument{b})}{} &  
constructeur par copie\\

\signature{const BodyEphemF\&  \fonc{operator =}}{(const BodyEphemF\& \argument{b})} & affectation \\
\signature{\fonc{\~{}BodyEphemF} ()}{} & destructeur \\

\signature{BodyEphem * \fonc{clone}}{() const} & permet la duplication \\

\signature{\fonc{BodyEphemF}}{(double \argument{equatorialRadius}, \\
double \argument{oblateness}, \\
double \argument{rotationVelocity}, \\
TypeFuncTsid *\argument{tsidFunc}, \\
TypeFuncPos *\argument{sunFunc}, \\
TypeFuncPos *\argument{moonFunc}, \\
TypeFuncPos *\argument{earthFunc})} & 
construit une instance � partir des caract�ristiques physiques du
corps central: rayon �quatorial \argument{equatorialRadius} (en~km),
aplatissement \argument{oblateness} et vitesse de
rotation \argument{rotationVelocity} (en~rad/s); et des fonctions de
calcul du temps sid�ral *\argument{tsidFunc} (en~rad), de calcul de la
position du Soleil *\argument{sunFunc} (en~km), de la Lune
*\argument{moonFunc} (en~km) et de la Terre *\argument{earthFunc}) (en~km)
\\
\hline

\signature{double  \fonc{siderealTime}}{(double \argument{t}, \\
double \argument{offset})} & 
calcule le temps sid�ral en \texttt{rad}
� la date~\argument{t} en \texttt{jour}, avec un �cart de datation
\argument{offset} (g�n�ralement en \texttt{s}) entre l'�chelle de
temps utilis� pour la date~\argument{t} et l'�chelle de temps utilis�
par le mod�le de calcul du temps sid�ral
\\

\signature{VecDBL  \fonc{sunPosition}}{(double \argument{t})} & 
calcule la position du Soleil par rapport au corps central en
\texttt{km} � la date~\argument{t}
\\

\signature{VecDBL  \fonc{moonPosition}}{(double \argument{t})} & 
calcule la position de la Lune par rapport au corps central en
\texttt{km} � la date~\argument{t} 
\\

\signature{VecDBL  \fonc{earthPosition}}{(double \argument{t})} & 
calcule la position de la Terre par rapport au corps central en
\texttt{km} � la date~\argument{t}
\\

\end{tableFonctionsFixe}
\subsubsection*{exemple d'utilisation}

L'exemple suivant, directement extrait du code de la biblioth�que
montre comment, dans la classe Etat, on initialise les donn�es et
fonctions donn�es par l'utilisateur, dans le cas du langage \langage{fortran}.


\begin{verbatim}
#include "marmottes/BodyEphemF.h"

    if (ptrBodyEphem_ != 0)
    {
      delete ptrBodyEphem_;
    }
    ptrBodyEphem_ = new BodyEphemF(equatorialRadius, oblateness, rotationVelocity, 
                                   tsidFunc, sunFunc, moonFunc, earthFunc);


\end{verbatim}

\subsubsection*{conseils d'utilisation
sp�cifiques}\label{sec:BodyEphemF-conseils}

Les unit�s utilis�es, les valeurs par d�faut, un rappel sur la notion
de date utilis�e dans \bibliotheque{marmottes} et des indications sur
la d�finition des fonctions par l'utilisateur sont donn�s au niveau de
la description de l'interface utilisateur (cf~\ref{sec:mod-ephem}).

\subsubsection*{implantation}\label{sec:BodyEphemF-impl}
Les attributs priv�s sont d�crits sommairement dans la
table~\ref{tab:BodyEphemF-att-priv}, il n'y a pas d'attribut prot�g�.
\begin{tableAttributsFixe}{attributs priv�s de la classe BodyEphemF}
{\label{tab:BodyEphemF-att-priv}}
{xxxxxxxxxxxxxxxxxxxxxxxxxxxxxxxxxxxxxxxxxxxxxxxxxxxxx}

tsidFuncPtr\_ & TypeFuncTsid * & pointeur sur la fonction de calcul du
temps sid�ral \\

sunFuncPtr\_ & TypeFuncPos * &  pointeur sur la fonction de calcul de
la position du Soleil \\

moonFuncPtr\_ & TypeFuncPos * &   pointeur sur la fonction de calcul de
la position de la Lune \\

earthFuncPtr\_ & TypeFuncPos * &  pointeur sur la fonction de calcul de
la position de la Terre \\

\end{tableAttributsFixe}
