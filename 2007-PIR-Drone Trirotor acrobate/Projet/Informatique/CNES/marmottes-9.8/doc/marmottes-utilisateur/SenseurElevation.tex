% -*- mode: latex; tex-main-file: "marmottes-utilisateur.tex" -*-
% $Id: SenseurElevation.tex,v 1.13 2004/06/21 14:43:11 marmottes Exp $
\subsection{classe SenseurElevation}\label{sec:SenseurElevation}

\subsubsection*{description}\label{sec:SenseurElevation-desc}

Cette classe implante les senseurs mesurant des angles d'�l�vation par
rapport � un plan de r�f�rence, elle permet d'obtenir les coordonn�es
sph�riques d'un astre en associant un SenseurDiedre et un SenseurElevation.
\subsubsection*{interface publique}\label{sec:SenseurElevation-int}
\begin{verbatim}
#include "marmottes/SenseurElevation.h"
\end{verbatim}

\begin{tableFonctionsFixe}{SenseurElevation : m�thodes publiques}
{\label{tab:SenseurElevation-met-pub}}
{construit une instance � partir des donn�es technologiques}

\signature{\fonc{SenseurElevation}}
          {(const string\& \argument{nom},\\
            const RotDBL\& \argument{repere},\\
            const VecDBL\& \argument{axeCalage},\\
            double \argument{precision},\\
            codeCible \argument{code},\\
            const StationCible *\argument{ptrStation},\\
            const VecDBL\& \argument{observe},\\
            Parcelle* \argument{ptrChampDeVue},\\
            Parcelle* \argument{ptrChampInhibitionSoleil},\\
            Parcelle* \argument{ptrChampInhibitionLune},\\
            Parcelle* \argument{ptrChampInhibitionCentral},\\
            double \argument{margeEclipseSoleil},\\
            double \argument{margeEclipseLune},\\
            double \argument{seuilPhaseLune},\\
            const VecDBL\& \argument{reference})
          }&

construit une instance � partir des donn�es technologiques\\

\hline

\signature{\fonc{SenseurElevation}}
          {(const SenseurElevation\& \argument{s})}&

constructeur par copie\\

\signature{SenseurElevation\& \fonc{operator =}}
          {(const SenseurElevation\& \argument{s})}&

affectation\\

\hline

\signature{\fonc{\~{}SenseurElevation} ()}
          {}&

destructeur, ne fait rien dans cette classe\\

\hline

\signature{Senseur* \fonc{copie} () const}
          {}&

op�rateur de copie virtuel\\

\signature{void \fonc{modeliseConsigne}}
          {(const Etat\& \argument{etat}, double \argument{valeur})\\
          \throw{CantorErreurs, MarmottesErreurs}}&

mod�lise la consigne \argument{valeur} dans l'\argument{etat} fourni\\

\signature{double \fonc{mesure} (const Etat\& \argument{etat})}
          {\throw{MarmottesErreurs}}&

retourne la mesure que produirait le senseur dans l'\argument{etat}
fourni\\

\end{tableFonctionsFixe}

\subsubsection*{implantation}\label{sec:SenseurElevation-impl}
Il n'y a ni attribut priv�, ni attribut prot�g�.

Les m�thodes prot�g�es sont d�crites dans la table~\ref{tab:SenseurElevation-met-prot}.
\begin{tableFonctionsFixe}{SenseurElevation : m�thodes prot�g�es}
{\label{tab:SenseurElevation-met-prot}}
{constructeur par d�faut. Il est d�fini explicitement uniquement pour }

\signature{\fonc{SenseurElevation} ()}
          {}&

constructeur par d�faut. Il est d�fini explicitement uniquement pour
pr�venir celui cr�� automatiquement par le compilateur et ne doit pas �tre
utilis�.
\\
\end{tableFonctionsFixe}
