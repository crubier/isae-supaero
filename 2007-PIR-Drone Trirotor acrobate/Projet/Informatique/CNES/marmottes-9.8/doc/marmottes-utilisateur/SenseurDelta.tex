% -*- mode: latex; tex-main-file: "marmottes-utilisateur.tex" -*-
% $Id: SenseurDelta.tex,v 1.7 2003/07/09 09:24:58 marmottes Exp $
\subsection{classe SenseurDelta}\label{sec:SenseurDelta}

\subsubsection*{description}\label{sec:SenseurDelta-desc}

Cette classe implante les pseudo-senseurs de d�clinaison, elle est
pratique par exemple pour optimiser une direction de pouss�e
inertielle (le vecteur interne de r�f�rence est alors la direction de
pouss�e).
\subsubsection*{interface publique}\label{sec:SenseurDelta-int}
\begin{verbatim}
#include "marmottes/SenseurDelta.h"
\end{verbatim}
\begin{tableFonctionsFixe}{SenseurDelta : m�thodes publiques}
{\label{tab:SenseurDelta-met-pub}}
{construit une instance � partir des donn�es technologiques}

\signature{\fonc{SenseurDelta}}
          {(const string\& \argument{nom},\\
            const RotDBL\& \argument{repere},\\
            const VecDBL\& \argument{axeCalage},\\
            double \argument{precision},\\
            const VecDBL\& \argument{observe})
          }&

construit une instance � partir des donn�es technologiques\\

\hline

\signature{\fonc{SenseurDelta} (const SenseurDelta\& \argument{s})}
          {}&

constructeur par copie\\

\signature{SenseurDelta\& \fonc{operator =}}
          {(const SenseurDelta\& \argument{s})}&

affectation\\

\hline

\signature{\fonc{\~{}SenseurDelta} ()}
          {}&

destructeur, ne fait rien dans cette classe\\

\hline

\signature{void \fonc{respecterMesures} ()}
          {}&

force le senseur � respecter les unit�s de mesures dans ses sorties\\

\signature{void \fonc{convertirMesures} ()}
          {}&

force le senseur � convertir les unit�s de mesures dans ses sorties\\

\hline

\signature{int \fonc{controlable} (const Etat\& \argument{etat})}
          {\throw{MarmottesErreurs}}&

indique si le senseur serait capable de contr�ler le satellite dans
l'\argument{etat} fourni (toujours vrai pour un senseur de
d�clinaison)\\

\hline

\signature{Senseur* \fonc{copie} () const}
          {}&

op�rateur de copie virtuel\\

\signature{typeGeom \fonc{typeGeometrique} () const}
          {}&

retourne le type de senseur g�om�trique (pseudoSenseur)\\

\hline

\signature{void \fonc{nouveauRepere}}
          {(const RotDBL\& \argument{nouveau})}&

remplace le rep�re du senseur par le \argument{nouveau}\\

\signature{void \fonc{modeliseConsigne}}
          {(const Etat\& \argument{etat},\\
            double \argument{valeur})\\
          \throw{CantorErreurs, MarmottesErreurs}}&

mod�lise la consigne \argument{valeur} dans l'\argument{etat} fourni\\

\signature{double \fonc{mesure} (const Etat\& \argument{etat})}
          {\throw{MarmottesErreurs}}&

retourne la mesure que produirait le senseur dans l'\argument{etat}
fourni\\

\end{tableFonctionsFixe}
\subsubsection*{implantation}\label{sec:SenseurDelta-impl}
Les attributs priv�s sont d�crits sommairement dans la
table~\ref{tab:SenseurDelta-att-priv}, il n'y a pas d'attribut prot�g�.
\begin{tableAttributsFixe}{attributs priv�s de la classe SenseurDelta}
{\label{tab:SenseurDelta-att-priv}}
{vecteur interne observ� (en rep�re satellite)}

observe\_ & VecDBL  & vecteur interne observ� (en rep�re satellite)\\

\end{tableAttributsFixe}

Les m�thodes prot�g�es sont d�crites dans la table~\ref{tab:SenseurDelta-met-prot}.
\begin{tableFonctionsFixe}{SenseurDelta : m�thodes prot�g�es}
{\label{tab:SenseurDelta-met-prot}}
{constructeur par d�faut. Il est d�fini explicitement uniquement pour }

\signature{\fonc{SenseurDelta} ()}
          {}&

constructeur par d�faut. Il est d�fini explicitement uniquement pour
pr�venir celui cr�� automatiquement par le compilateur et ne doit pas �tre
utilis�.
\\
\end{tableFonctionsFixe}
