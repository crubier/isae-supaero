% -*- mode: latex; tex-main-file: "marmottes-utilisateur.tex" -*-
% $Id: SenseurFonctionSinCard2.tex,v 1.9 2004/06/21 14:43:17 marmottes Exp $
\subsection{classe SenseurFonctionSinCard2}\label{sec:SenseurFonctionSinCard2}

\subsubsection*{description}\label{sec:SenseurFonctionSinCard2-desc}

Cette classe implante les pseudo-senseurs mesurant des gains
d'antennes bord dans une direction particuli�re d�finie par la cible
(typiquement une station sol). Ce type de senseur permet de calculer
la part li�e uniquement � l'attitude dans un bilan de liaison. La
forme du gain mod�lis�e par ce senseur est :
\begin{displaymath}
g = 10\times\frac{\log{K \left( \frac{\sin\theta}{\theta} \right) ^2}}
                 {\log{10}}
\end{displaymath}
o� $\theta$ est le d�pointage entre l'axe d'antenne et la cible. La
valeur retourn�e est donc directement une valeur en dB.

La fonction n'�tant pas inversible (� moins de se restreindre au lobe
primaire), on ne \emph{peut pas utiliser ces senseurs en consigne},
ils sont destin�s � �tre utilis�s en \emph{mesure uniquement}.

\subsubsection*{interface publique}\label{sec:SenseurFonctionSinCard2-int}
\begin{verbatim}
#include "marmottes/SenseurFonctionSinCard2.h"
\end{verbatim}

\begin{tableFonctionsFixe}{SenseurFonctionSinCard2 : m�thodes publiques}
{\label{tab:SenseurFonctionSinCard2-met-pub}}
{construit une instance � partir des donn�es technologiques}

\signature{\fonc{SenseurFonctionSinCard2}}
          {(const string\& \argument{nom},\\
            const RotDBL\& \argument{repere},\\
            const VecDBL\& \argument{axeCalage},\\
            double \argument{precision},\\
            codeCible \argument{code},\\
            const StationCible *\argument{ptrStation},\\
            const VecDBL\& \argument{observe},\\
            Parcelle* \argument{ptrChampDeVue},\\
            Parcelle* \argument{ptrChampInhibitionSoleil},\\
            Parcelle* \argument{ptrChampInhibitionLune},\\
            Parcelle* \argument{ptrChampInhibitionCentral},\\
            double \argument{margeEclipseSoleil},\\
            double \argument{margeEclipseLune},\\
            double \argument{seuilPhaseLune},\\
            const VecDBL\& \argument{axe},\\
            const VecDBL\& \argument{origine},\\
            double \argument{maximum},\\
            double \argument{angle3dB})\\
            \throw{CantorErreurs}} &
construit une instance � partir des donn�es technologiques\\

\hline

\signature{\fonc{SenseurFonctionSinCard2}}{(const SenseurFonctionSinCard2\& \argument{s})} & constructeur par copie\\

\signature{SenseurFonctionSinCard2\&  \fonc{operator =}}{(const SenseurFonctionSinCard2\& \argument{s})} & affectation\\

\hline

\signature{\fonc{\~{}SenseurFonctionSinCard2} ()}{} & destructeur, ne
fait rien dans cette classe\\

\hline

\signature{Senseur*  \fonc{copie} () const}{} & op�rateur de copie virtuel\\

\signature{double \fonc{fonction}}
          {(double \argument{azimut}, double \argument{depointage}) const\\
           \throw{MarmottesErreurs}
          } &

cette m�thode �value la fonction de gain $$g = 10\times\frac{\log{K
\left( \frac{\sin\theta}{\theta} \right) ^2}}{\log{10}}$$ o� $\theta$
est le \argument{depointage}\\
\end{tableFonctionsFixe}

\subsubsection*{implantation}\label{sec:SenseurFonctionSinCard2-impl}
Les attributs priv�s sont d�crits sommairement dans la
table~\ref{tab:SenseurFonctionSinCard2-att-priv}, il n'y a pas d'attribut prot�g�.
\begin{tableAttributsFixe}{attributs priv�s de la classe SenseurFonctionSinCard2}
{\label{tab:SenseurFonctionSinCard2-att-priv}}
{valeur maximale du gain dans la direction de l'axe}

maximum\_ & double & valeur maximale en dB du gain dans la direction
de l'axe\\

angle3dB\_ & double & angle de d�pointage pour lequel le gain chute de
3 dB (c'est � dire que la fonction dans le logarithme est divis�e par deux)\\

\end{tableAttributsFixe}

Les m�thodes prot�g�es sont d�crites dans la table~\ref{tab:SenseurFonctionSinCard2-met-prot}.
\begin{tableFonctionsFixe}{SenseurFonctionSinCard2 : m�thodes prot�g�es}
{\label{tab:SenseurFonctionSinCard2-met-prot}}
{constructeur par d�faut. Il est d�fini explicitement uniquement pour }

\signature{\fonc{SenseurFonctionSinCard2} ()}
          {}&

constructeur par d�faut. Il est d�fini explicitement uniquement pour
pr�venir celui cr�� automatiquement par le compilateur et ne doit pas �tre
utilis�.
\\
\end{tableFonctionsFixe}
